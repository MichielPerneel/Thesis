\documentclass[]{article}
\usepackage{lmodern}
\usepackage{amssymb,amsmath}
\usepackage{ifxetex,ifluatex}
\usepackage{fixltx2e} % provides \textsubscript
\ifnum 0\ifxetex 1\fi\ifluatex 1\fi=0 % if pdftex
  \usepackage[T1]{fontenc}
  \usepackage[utf8]{inputenc}
\else % if luatex or xelatex
  \ifxetex
    \usepackage{mathspec}
  \else
    \usepackage{fontspec}
  \fi
  \defaultfontfeatures{Ligatures=TeX,Scale=MatchLowercase}
\fi
% use upquote if available, for straight quotes in verbatim environments
\IfFileExists{upquote.sty}{\usepackage{upquote}}{}
% use microtype if available
\IfFileExists{microtype.sty}{%
\usepackage{microtype}
\UseMicrotypeSet[protrusion]{basicmath} % disable protrusion for tt fonts
}{}
\usepackage[margin=1in]{geometry}
\usepackage{hyperref}
\hypersetup{unicode=true,
            pdftitle={Data exploration Thesis},
            pdfauthor={Michiel Perneel},
            pdfborder={0 0 0},
            breaklinks=true}
\urlstyle{same}  % don't use monospace font for urls
\usepackage{color}
\usepackage{fancyvrb}
\newcommand{\VerbBar}{|}
\newcommand{\VERB}{\Verb[commandchars=\\\{\}]}
\DefineVerbatimEnvironment{Highlighting}{Verbatim}{commandchars=\\\{\}}
% Add ',fontsize=\small' for more characters per line
\usepackage{framed}
\definecolor{shadecolor}{RGB}{248,248,248}
\newenvironment{Shaded}{\begin{snugshade}}{\end{snugshade}}
\newcommand{\KeywordTok}[1]{\textcolor[rgb]{0.13,0.29,0.53}{\textbf{#1}}}
\newcommand{\DataTypeTok}[1]{\textcolor[rgb]{0.13,0.29,0.53}{#1}}
\newcommand{\DecValTok}[1]{\textcolor[rgb]{0.00,0.00,0.81}{#1}}
\newcommand{\BaseNTok}[1]{\textcolor[rgb]{0.00,0.00,0.81}{#1}}
\newcommand{\FloatTok}[1]{\textcolor[rgb]{0.00,0.00,0.81}{#1}}
\newcommand{\ConstantTok}[1]{\textcolor[rgb]{0.00,0.00,0.00}{#1}}
\newcommand{\CharTok}[1]{\textcolor[rgb]{0.31,0.60,0.02}{#1}}
\newcommand{\SpecialCharTok}[1]{\textcolor[rgb]{0.00,0.00,0.00}{#1}}
\newcommand{\StringTok}[1]{\textcolor[rgb]{0.31,0.60,0.02}{#1}}
\newcommand{\VerbatimStringTok}[1]{\textcolor[rgb]{0.31,0.60,0.02}{#1}}
\newcommand{\SpecialStringTok}[1]{\textcolor[rgb]{0.31,0.60,0.02}{#1}}
\newcommand{\ImportTok}[1]{#1}
\newcommand{\CommentTok}[1]{\textcolor[rgb]{0.56,0.35,0.01}{\textit{#1}}}
\newcommand{\DocumentationTok}[1]{\textcolor[rgb]{0.56,0.35,0.01}{\textbf{\textit{#1}}}}
\newcommand{\AnnotationTok}[1]{\textcolor[rgb]{0.56,0.35,0.01}{\textbf{\textit{#1}}}}
\newcommand{\CommentVarTok}[1]{\textcolor[rgb]{0.56,0.35,0.01}{\textbf{\textit{#1}}}}
\newcommand{\OtherTok}[1]{\textcolor[rgb]{0.56,0.35,0.01}{#1}}
\newcommand{\FunctionTok}[1]{\textcolor[rgb]{0.00,0.00,0.00}{#1}}
\newcommand{\VariableTok}[1]{\textcolor[rgb]{0.00,0.00,0.00}{#1}}
\newcommand{\ControlFlowTok}[1]{\textcolor[rgb]{0.13,0.29,0.53}{\textbf{#1}}}
\newcommand{\OperatorTok}[1]{\textcolor[rgb]{0.81,0.36,0.00}{\textbf{#1}}}
\newcommand{\BuiltInTok}[1]{#1}
\newcommand{\ExtensionTok}[1]{#1}
\newcommand{\PreprocessorTok}[1]{\textcolor[rgb]{0.56,0.35,0.01}{\textit{#1}}}
\newcommand{\AttributeTok}[1]{\textcolor[rgb]{0.77,0.63,0.00}{#1}}
\newcommand{\RegionMarkerTok}[1]{#1}
\newcommand{\InformationTok}[1]{\textcolor[rgb]{0.56,0.35,0.01}{\textbf{\textit{#1}}}}
\newcommand{\WarningTok}[1]{\textcolor[rgb]{0.56,0.35,0.01}{\textbf{\textit{#1}}}}
\newcommand{\AlertTok}[1]{\textcolor[rgb]{0.94,0.16,0.16}{#1}}
\newcommand{\ErrorTok}[1]{\textcolor[rgb]{0.64,0.00,0.00}{\textbf{#1}}}
\newcommand{\NormalTok}[1]{#1}
\usepackage{graphicx,grffile}
\makeatletter
\def\maxwidth{\ifdim\Gin@nat@width>\linewidth\linewidth\else\Gin@nat@width\fi}
\def\maxheight{\ifdim\Gin@nat@height>\textheight\textheight\else\Gin@nat@height\fi}
\makeatother
% Scale images if necessary, so that they will not overflow the page
% margins by default, and it is still possible to overwrite the defaults
% using explicit options in \includegraphics[width, height, ...]{}
\setkeys{Gin}{width=\maxwidth,height=\maxheight,keepaspectratio}
\IfFileExists{parskip.sty}{%
\usepackage{parskip}
}{% else
\setlength{\parindent}{0pt}
\setlength{\parskip}{6pt plus 2pt minus 1pt}
}
\setlength{\emergencystretch}{3em}  % prevent overfull lines
\providecommand{\tightlist}{%
  \setlength{\itemsep}{0pt}\setlength{\parskip}{0pt}}
\setcounter{secnumdepth}{0}
% Redefines (sub)paragraphs to behave more like sections
\ifx\paragraph\undefined\else
\let\oldparagraph\paragraph
\renewcommand{\paragraph}[1]{\oldparagraph{#1}\mbox{}}
\fi
\ifx\subparagraph\undefined\else
\let\oldsubparagraph\subparagraph
\renewcommand{\subparagraph}[1]{\oldsubparagraph{#1}\mbox{}}
\fi

%%% Use protect on footnotes to avoid problems with footnotes in titles
\let\rmarkdownfootnote\footnote%
\def\footnote{\protect\rmarkdownfootnote}

%%% Change title format to be more compact
\usepackage{titling}

% Create subtitle command for use in maketitle
\newcommand{\subtitle}[1]{
  \posttitle{
    \begin{center}\large#1\end{center}
    }
}

\setlength{\droptitle}{-2em}

  \title{Data exploration Thesis}
    \pretitle{\vspace{\droptitle}\centering\huge}
  \posttitle{\par}
    \author{Michiel Perneel}
    \preauthor{\centering\large\emph}
  \postauthor{\par}
      \predate{\centering\large\emph}
  \postdate{\par}
    \date{March 2019}


\begin{document}
\maketitle

\section{Data setup}\label{data-setup}

In this section, all the required libraries are read, and the data is
uploaded in r.
\url{https://bookdown.org/yihui/rmarkdown/output-formats.html}

Now the data is loaded

\begin{Shaded}
\begin{Highlighting}[]
\KeywordTok{setwd}\NormalTok{(}\StringTok{"C:/Users/user/Desktop/Biologie/Master/Thesis/Thesis/data/raw"}\NormalTok{)}
\NormalTok{maaginhoud <-}\StringTok{ }\KeywordTok{as.tibble}\NormalTok{(}\KeywordTok{read.csv2}\NormalTok{(}\StringTok{"C:/Users/user/Desktop/Biologie/Master/Thesis/Thesis/data/raw/Maaginhoud.csv"}\NormalTok{, }\DataTypeTok{sep =} \StringTok{";"}\NormalTok{, }\DataTypeTok{dec =} \StringTok{","}\NormalTok{)) }\CommentTok{# leest de ruwe dataset (count van maaginhoud)}
\end{Highlighting}
\end{Shaded}

\begin{verbatim}
## Warning: `as.tibble()` is deprecated, use `as_tibble()` (but mind the new semantics).
## This warning is displayed once per session.
\end{verbatim}

\begin{Shaded}
\begin{Highlighting}[]
\KeywordTok{head}\NormalTok{(maaginhoud)}
\end{Highlighting}
\end{Shaded}

\begin{verbatim}
## # A tibble: 6 x 77
##      nr Potje.staart.ve~ Potje.Maaginhoud datum Locatie Gewicht.Glasaal
##   <int> <fct>                       <int> <fct> <fct>             <dbl>
## 1    60 A                               1 24/0~ PG-KR             0.211
## 2   278 A                               1 12/0~ MI-RO-~           0.294
## 3   215 A                               1 5/05~ MI-RO-~           0.211
## 4   183 A                               1 28/0~ PG-LO-~           0.147
## 5   219 A                               1 12/0~ PG-RO-~           0.227
## 6   242 A                               1 15/0~ PG-LO-~           0.2  
## # ... with 71 more variables: Gewicht.maag.darmstelsel <dbl>,
## #   Opmerkingen <fct>, Microplastics.Contaminatie.vezels..katoen.. <int>,
## #   X <lgl>, Cladocera.sp. <int>, X.1 <lgl>, Annelida <int>, X.2 <lgl>,
## #   Polychaete <int>, X.3 <lgl>, Isopoda <int>, X.4 <lgl>,
## #   Diptera.sp. <int>, X.5 <lgl>, Chironomide.thummi.plummosus <int>,
## #   X.6 <lgl>, Crustaceae.sp. <int>, X.7 <lgl>, Cycloida.sp. <int>,
## #   X.8 <lgl>, Invertebrata..larve..sp. <int>, X.9 <lgl>,
## #   Daphnidae.sp...ingekapseld. <int>, X.10 <lgl>,
## #   Cladocera.sp...Chydorus.sphaericus. <int>, X.11 <lgl>,
## #   Amphipoda.sp. <int>, X.12 <lgl>,
## #   Calanoide.Copepode..Pseudocalanus.elongatus. <int>, X.13 <lgl>,
## #   Gammarus <int>, X.14 <lgl>, Calanoide.copepode <int>, X.15 <lgl>,
## #   unidentifiable.eggs <int>, X.16 <lgl>, Copepoda.sp. <int>, X.17 <lgl>,
## #   Copepoda..Oithona.similis..male. <int>, X.18 <lgl>,
## #   Harpacticoid.Copepod <int>, X.19 <lgl>, Pleuroxus.sp. <int>,
## #   X.20 <lgl>, Chydorus.sp. <int>, X.21 <lgl>,
## #   Detritus.plantenafval <int>, X.22 <lgl>, Plantae.sp. <int>,
## #   X.23 <lgl>, Pennate.diatomee <int>, X.24 <lgl>,
## #   Detritus.met.schimmelfilamenten <int>, X.25 <lgl>,
## #   Invertebrate.sp. <int>, X.26 <lgl>, Hexapode.sp.... <int>, X.27 <lgl>,
## #   keverlarve <int>, X.28 <lgl>, Chironomide..Non.thummi.plumosus. <int>,
## #   X.29 <lgl>, Nereis.sp. <int>, X.30 <lgl>,
## #   Amphipode..Atylus.sp.. <int>, X.31 <lgl>, Chironomide.sp. <int>,
## #   X.32 <lgl>, Simulium.sp. <int>, X.33 <lgl>, X.34 <lgl>
\end{verbatim}

\begin{Shaded}
\begin{Highlighting}[]
\NormalTok{maaginhoud2 <-}\StringTok{ }\KeywordTok{as.tibble}\NormalTok{(}\KeywordTok{read.csv2}\NormalTok{(}\StringTok{"C:/Users/user/Desktop/Biologie/Master/Thesis/Thesis/data/raw/Maaginhoud2.csv"}\NormalTok{, }\DataTypeTok{sep =} \StringTok{";"}\NormalTok{, }\DataTypeTok{dec =} \StringTok{","}\NormalTok{))}
\KeywordTok{head}\NormalTok{(maaginhoud2)}
\end{Highlighting}
\end{Shaded}

\begin{verbatim}
## # A tibble: 6 x 24
##      nr datum Locatie Locatie_A Locatie_Oever Methode Microplastics.C~
##   <int> <fct> <fct>   <fct>     <fct>         <fct>              <int>
## 1    60 24/0~ PG-KR   Pompgema~ midden        Kruisn~                1
## 2   278 12/0~ MI-RO-~ Midden VA RO            Substr~                0
## 3   215 5/05~ MI-RO-~ Midden VA RO            Substr~                0
## 4   183 28/0~ PG-LO-~ Pompgema~ LO            Substr~                1
## 5   219 12/0~ PG-RO-~ Pompgema~ RO            Paling~                0
## 6   242 15/0~ PG-LO-~ Pompgema~ LO            Paling~                0
## # ... with 17 more variables: Cladocera.sp. <int>, Annelida <int>,
## #   Polychaete <int>, Isopoda <int>, Diptera.sp. <int>,
## #   Chironomida.sp. <int>, Crustaceae.sp. <int>, Cycloida.sp. <int>,
## #   Unidentified.sp. <int>, Amphipoda.sp. <int>, Calanoida.sp. <int>,
## #   Copepoda.sp. <int>, Harpacticoid.Copepod <int>, Hexapode.sp.... <int>,
## #   keverlarve <int>, Pennate.diatomee <int>, Plantae.sp. <int>
\end{verbatim}

\begin{Shaded}
\begin{Highlighting}[]
\NormalTok{data_inbo <-}\StringTok{ }\KeywordTok{as.tibble}\NormalTok{(}\KeywordTok{read.csv2}\NormalTok{(}\StringTok{"C:/Users/user/Desktop/Biologie/Master/Thesis/Thesis/data/raw/Glass eel stock deepfreeze 2017.csv"}\NormalTok{, }\DataTypeTok{sep =} \StringTok{";"}\NormalTok{, }\DataTypeTok{dec =} \StringTok{","}\NormalTok{)) }\CommentTok{# leest de data van het inbo genoteerd bij de glasaalstaalname}
\KeywordTok{head}\NormalTok{(data_inbo)}
\end{Highlighting}
\end{Shaded}

\begin{verbatim}
## # A tibble: 6 x 22
##      nr Date  Location Catchment.method length..mm. weight.mg.
##   <int> <fct> <fct>    <fct>                  <int>      <int>
## 1     1 10/0~ VA_PG_R~ Palinggoot                77        347
## 2     2 10/0~ VA_PG_R~ Palinggoot                74        315
## 3     3 10/0~ VA_PG_R~ Palinggoot                70        265
## 4     4 10/0~ VA_PG_R~ Palinggoot                68        266
## 5     5 10/0~ VA_PG_R~ Palinggoot                76        356
## 6     6 11/0~ VA_PG_L~ Palinggoot                66        313
## # ... with 16 more variables: pigmentation.stage <fct>,
## #   condition.factor <dbl>, Food.visible <fct>, remarks <fct>, X <lgl>,
## #   X.1 <fct>, X.2 <lgl>, X.3 <lgl>, X.4 <lgl>, X.5 <fct>, X.6 <lgl>,
## #   X.7 <lgl>, X.8 <lgl>, X.9 <lgl>, X.10 <lgl>, X.11 <lgl>
\end{verbatim}

\begin{Shaded}
\begin{Highlighting}[]
\NormalTok{data_inbo}\OperatorTok{$}\NormalTok{Date <-}\StringTok{ }\KeywordTok{strptime}\NormalTok{(}\KeywordTok{as.character}\NormalTok{(data_inbo}\OperatorTok{$}\NormalTok{Date), }\StringTok{"%d/%m/%Y"}\NormalTok{)}
\end{Highlighting}
\end{Shaded}

And then we can start exploring!

\section{Visual Exploration}\label{visual-exploration}

\subsection{Environmental Data from
INBO}\label{environmental-data-from-inbo}

\includegraphics{Data_Exploration_Thesis_files/figure-latex/pressure-1.pdf}

\begin{verbatim}
## List of 60
##  $ line                 :List of 6
##   ..$ colour       : chr "black"
##   ..$ size         : num 0.5
##   ..$ linetype     : num 1
##   ..$ lineend      : chr "butt"
##   ..$ arrow        : logi FALSE
##   ..$ inherit.blank: logi TRUE
##   ..- attr(*, "class")= chr [1:2] "element_line" "element"
##  $ rect                 :List of 5
##   ..$ fill         : chr "white"
##   ..$ colour       : chr "black"
##   ..$ size         : num 0.5
##   ..$ linetype     : num 1
##   ..$ inherit.blank: logi TRUE
##   ..- attr(*, "class")= chr [1:2] "element_rect" "element"
##  $ text                 :List of 11
##   ..$ family       : chr ""
##   ..$ face         : chr "plain"
##   ..$ colour       : chr "black"
##   ..$ size         : num 11
##   ..$ hjust        : num 0.5
##   ..$ vjust        : num 0.5
##   ..$ angle        : num 0
##   ..$ lineheight   : num 0.9
##   ..$ margin       :Classes 'margin', 'unit'  atomic [1:4] 0 0 0 0
##   .. .. ..- attr(*, "valid.unit")= int 8
##   .. .. ..- attr(*, "unit")= chr "pt"
##   ..$ debug        : logi FALSE
##   ..$ inherit.blank: logi TRUE
##   ..- attr(*, "class")= chr [1:2] "element_text" "element"
##  $ axis.title.x         :List of 11
##   ..$ family       : NULL
##   ..$ face         : NULL
##   ..$ colour       : NULL
##   ..$ size         : NULL
##   ..$ hjust        : NULL
##   ..$ vjust        : num 1
##   ..$ angle        : NULL
##   ..$ lineheight   : NULL
##   ..$ margin       :Classes 'margin', 'unit'  atomic [1:4] 2.75 0 0 0
##   .. .. ..- attr(*, "valid.unit")= int 8
##   .. .. ..- attr(*, "unit")= chr "pt"
##   ..$ debug        : NULL
##   ..$ inherit.blank: logi TRUE
##   ..- attr(*, "class")= chr [1:2] "element_text" "element"
##  $ axis.title.x.top     :List of 11
##   ..$ family       : NULL
##   ..$ face         : NULL
##   ..$ colour       : NULL
##   ..$ size         : NULL
##   ..$ hjust        : NULL
##   ..$ vjust        : num 0
##   ..$ angle        : NULL
##   ..$ lineheight   : NULL
##   ..$ margin       :Classes 'margin', 'unit'  atomic [1:4] 0 0 2.75 0
##   .. .. ..- attr(*, "valid.unit")= int 8
##   .. .. ..- attr(*, "unit")= chr "pt"
##   ..$ debug        : NULL
##   ..$ inherit.blank: logi TRUE
##   ..- attr(*, "class")= chr [1:2] "element_text" "element"
##  $ axis.title.y         :List of 11
##   ..$ family       : NULL
##   ..$ face         : NULL
##   ..$ colour       : NULL
##   ..$ size         : NULL
##   ..$ hjust        : NULL
##   ..$ vjust        : num 1
##   ..$ angle        : num 90
##   ..$ lineheight   : NULL
##   ..$ margin       :Classes 'margin', 'unit'  atomic [1:4] 0 2.75 0 0
##   .. .. ..- attr(*, "valid.unit")= int 8
##   .. .. ..- attr(*, "unit")= chr "pt"
##   ..$ debug        : NULL
##   ..$ inherit.blank: logi TRUE
##   ..- attr(*, "class")= chr [1:2] "element_text" "element"
##  $ axis.title.y.right   :List of 11
##   ..$ family       : NULL
##   ..$ face         : NULL
##   ..$ colour       : NULL
##   ..$ size         : NULL
##   ..$ hjust        : NULL
##   ..$ vjust        : num 0
##   ..$ angle        : num -90
##   ..$ lineheight   : NULL
##   ..$ margin       :Classes 'margin', 'unit'  atomic [1:4] 0 0 0 2.75
##   .. .. ..- attr(*, "valid.unit")= int 8
##   .. .. ..- attr(*, "unit")= chr "pt"
##   ..$ debug        : NULL
##   ..$ inherit.blank: logi TRUE
##   ..- attr(*, "class")= chr [1:2] "element_text" "element"
##  $ axis.text            :List of 11
##   ..$ family       : NULL
##   ..$ face         : NULL
##   ..$ colour       : chr "grey30"
##   ..$ size         :Class 'rel'  num 0.8
##   ..$ hjust        : NULL
##   ..$ vjust        : NULL
##   ..$ angle        : NULL
##   ..$ lineheight   : NULL
##   ..$ margin       : NULL
##   ..$ debug        : NULL
##   ..$ inherit.blank: logi TRUE
##   ..- attr(*, "class")= chr [1:2] "element_text" "element"
##  $ axis.text.x          :List of 11
##   ..$ family       : NULL
##   ..$ face         : NULL
##   ..$ colour       : NULL
##   ..$ size         : NULL
##   ..$ hjust        : NULL
##   ..$ vjust        : num 1
##   ..$ angle        : NULL
##   ..$ lineheight   : NULL
##   ..$ margin       :Classes 'margin', 'unit'  atomic [1:4] 2.2 0 0 0
##   .. .. ..- attr(*, "valid.unit")= int 8
##   .. .. ..- attr(*, "unit")= chr "pt"
##   ..$ debug        : NULL
##   ..$ inherit.blank: logi TRUE
##   ..- attr(*, "class")= chr [1:2] "element_text" "element"
##  $ axis.text.x.top      :List of 11
##   ..$ family       : NULL
##   ..$ face         : NULL
##   ..$ colour       : NULL
##   ..$ size         : NULL
##   ..$ hjust        : NULL
##   ..$ vjust        : num 0
##   ..$ angle        : NULL
##   ..$ lineheight   : NULL
##   ..$ margin       :Classes 'margin', 'unit'  atomic [1:4] 0 0 2.2 0
##   .. .. ..- attr(*, "valid.unit")= int 8
##   .. .. ..- attr(*, "unit")= chr "pt"
##   ..$ debug        : NULL
##   ..$ inherit.blank: logi TRUE
##   ..- attr(*, "class")= chr [1:2] "element_text" "element"
##  $ axis.text.y          :List of 11
##   ..$ family       : NULL
##   ..$ face         : NULL
##   ..$ colour       : NULL
##   ..$ size         : NULL
##   ..$ hjust        : num 1
##   ..$ vjust        : NULL
##   ..$ angle        : NULL
##   ..$ lineheight   : NULL
##   ..$ margin       :Classes 'margin', 'unit'  atomic [1:4] 0 2.2 0 0
##   .. .. ..- attr(*, "valid.unit")= int 8
##   .. .. ..- attr(*, "unit")= chr "pt"
##   ..$ debug        : NULL
##   ..$ inherit.blank: logi TRUE
##   ..- attr(*, "class")= chr [1:2] "element_text" "element"
##  $ axis.text.y.right    :List of 11
##   ..$ family       : NULL
##   ..$ face         : NULL
##   ..$ colour       : NULL
##   ..$ size         : NULL
##   ..$ hjust        : num 0
##   ..$ vjust        : NULL
##   ..$ angle        : NULL
##   ..$ lineheight   : NULL
##   ..$ margin       :Classes 'margin', 'unit'  atomic [1:4] 0 0 0 2.2
##   .. .. ..- attr(*, "valid.unit")= int 8
##   .. .. ..- attr(*, "unit")= chr "pt"
##   ..$ debug        : NULL
##   ..$ inherit.blank: logi TRUE
##   ..- attr(*, "class")= chr [1:2] "element_text" "element"
##  $ axis.ticks           :List of 6
##   ..$ colour       : chr "grey20"
##   ..$ size         : NULL
##   ..$ linetype     : NULL
##   ..$ lineend      : NULL
##   ..$ arrow        : logi FALSE
##   ..$ inherit.blank: logi TRUE
##   ..- attr(*, "class")= chr [1:2] "element_line" "element"
##  $ axis.ticks.length    :Class 'unit'  atomic [1:1] 2.75
##   .. ..- attr(*, "valid.unit")= int 8
##   .. ..- attr(*, "unit")= chr "pt"
##  $ axis.line            :List of 6
##   ..$ colour       : chr "black"
##   ..$ size         :Class 'rel'  num 1
##   ..$ linetype     : NULL
##   ..$ lineend      : NULL
##   ..$ arrow        : logi FALSE
##   ..$ inherit.blank: logi TRUE
##   ..- attr(*, "class")= chr [1:2] "element_line" "element"
##  $ axis.line.x          : NULL
##  $ axis.line.y          : NULL
##  $ legend.background    :List of 5
##   ..$ fill         : NULL
##   ..$ colour       : logi NA
##   ..$ size         : NULL
##   ..$ linetype     : NULL
##   ..$ inherit.blank: logi TRUE
##   ..- attr(*, "class")= chr [1:2] "element_rect" "element"
##  $ legend.margin        :Classes 'margin', 'unit'  atomic [1:4] 5.5 5.5 5.5 5.5
##   .. ..- attr(*, "valid.unit")= int 8
##   .. ..- attr(*, "unit")= chr "pt"
##  $ legend.spacing       :Class 'unit'  atomic [1:1] 11
##   .. ..- attr(*, "valid.unit")= int 8
##   .. ..- attr(*, "unit")= chr "pt"
##  $ legend.spacing.x     : NULL
##  $ legend.spacing.y     : NULL
##  $ legend.key           : list()
##   ..- attr(*, "class")= chr [1:2] "element_blank" "element"
##  $ legend.key.size      :Class 'unit'  atomic [1:1] 1.2
##   .. ..- attr(*, "valid.unit")= int 3
##   .. ..- attr(*, "unit")= chr "lines"
##  $ legend.key.height    : NULL
##  $ legend.key.width     : NULL
##  $ legend.text          :List of 11
##   ..$ family       : NULL
##   ..$ face         : NULL
##   ..$ colour       : NULL
##   ..$ size         :Class 'rel'  num 0.8
##   ..$ hjust        : NULL
##   ..$ vjust        : NULL
##   ..$ angle        : NULL
##   ..$ lineheight   : NULL
##   ..$ margin       : NULL
##   ..$ debug        : NULL
##   ..$ inherit.blank: logi TRUE
##   ..- attr(*, "class")= chr [1:2] "element_text" "element"
##  $ legend.text.align    : NULL
##  $ legend.title         :List of 11
##   ..$ family       : NULL
##   ..$ face         : NULL
##   ..$ colour       : NULL
##   ..$ size         : NULL
##   ..$ hjust        : num 0
##   ..$ vjust        : NULL
##   ..$ angle        : NULL
##   ..$ lineheight   : NULL
##   ..$ margin       : NULL
##   ..$ debug        : NULL
##   ..$ inherit.blank: logi TRUE
##   ..- attr(*, "class")= chr [1:2] "element_text" "element"
##  $ legend.title.align   : NULL
##  $ legend.position      : chr "right"
##  $ legend.direction     : NULL
##  $ legend.justification : chr "center"
##  $ legend.box           : NULL
##  $ legend.box.margin    :Classes 'margin', 'unit'  atomic [1:4] 0 0 0 0
##   .. ..- attr(*, "valid.unit")= int 1
##   .. ..- attr(*, "unit")= chr "cm"
##  $ legend.box.background: list()
##   ..- attr(*, "class")= chr [1:2] "element_blank" "element"
##  $ legend.box.spacing   :Class 'unit'  atomic [1:1] 11
##   .. ..- attr(*, "valid.unit")= int 8
##   .. ..- attr(*, "unit")= chr "pt"
##  $ panel.background     :List of 5
##   ..$ fill         : chr "white"
##   ..$ colour       : logi NA
##   ..$ size         : NULL
##   ..$ linetype     : NULL
##   ..$ inherit.blank: logi TRUE
##   ..- attr(*, "class")= chr [1:2] "element_rect" "element"
##  $ panel.border         : list()
##   ..- attr(*, "class")= chr [1:2] "element_blank" "element"
##  $ panel.spacing        :Class 'unit'  atomic [1:1] 5.5
##   .. ..- attr(*, "valid.unit")= int 8
##   .. ..- attr(*, "unit")= chr "pt"
##  $ panel.spacing.x      : NULL
##  $ panel.spacing.y      : NULL
##  $ panel.grid           :List of 6
##   ..$ colour       : chr "grey92"
##   ..$ size         : NULL
##   ..$ linetype     : NULL
##   ..$ lineend      : NULL
##   ..$ arrow        : logi FALSE
##   ..$ inherit.blank: logi TRUE
##   ..- attr(*, "class")= chr [1:2] "element_line" "element"
##  $ panel.grid.minor     : list()
##   ..- attr(*, "class")= chr [1:2] "element_blank" "element"
##  $ panel.ontop          : logi FALSE
##  $ plot.background      :List of 5
##   ..$ fill         : NULL
##   ..$ colour       : chr "white"
##   ..$ size         : NULL
##   ..$ linetype     : NULL
##   ..$ inherit.blank: logi TRUE
##   ..- attr(*, "class")= chr [1:2] "element_rect" "element"
##  $ plot.title           :List of 11
##   ..$ family       : NULL
##   ..$ face         : NULL
##   ..$ colour       : NULL
##   ..$ size         :Class 'rel'  num 1.2
##   ..$ hjust        : num 0
##   ..$ vjust        : num 1
##   ..$ angle        : NULL
##   ..$ lineheight   : NULL
##   ..$ margin       :Classes 'margin', 'unit'  atomic [1:4] 0 0 5.5 0
##   .. .. ..- attr(*, "valid.unit")= int 8
##   .. .. ..- attr(*, "unit")= chr "pt"
##   ..$ debug        : NULL
##   ..$ inherit.blank: logi TRUE
##   ..- attr(*, "class")= chr [1:2] "element_text" "element"
##  $ plot.subtitle        :List of 11
##   ..$ family       : NULL
##   ..$ face         : NULL
##   ..$ colour       : NULL
##   ..$ size         : NULL
##   ..$ hjust        : num 0
##   ..$ vjust        : num 1
##   ..$ angle        : NULL
##   ..$ lineheight   : NULL
##   ..$ margin       :Classes 'margin', 'unit'  atomic [1:4] 0 0 5.5 0
##   .. .. ..- attr(*, "valid.unit")= int 8
##   .. .. ..- attr(*, "unit")= chr "pt"
##   ..$ debug        : NULL
##   ..$ inherit.blank: logi TRUE
##   ..- attr(*, "class")= chr [1:2] "element_text" "element"
##  $ plot.caption         :List of 11
##   ..$ family       : NULL
##   ..$ face         : NULL
##   ..$ colour       : NULL
##   ..$ size         :Class 'rel'  num 0.8
##   ..$ hjust        : num 1
##   ..$ vjust        : num 1
##   ..$ angle        : NULL
##   ..$ lineheight   : NULL
##   ..$ margin       :Classes 'margin', 'unit'  atomic [1:4] 5.5 0 0 0
##   .. .. ..- attr(*, "valid.unit")= int 8
##   .. .. ..- attr(*, "unit")= chr "pt"
##   ..$ debug        : NULL
##   ..$ inherit.blank: logi TRUE
##   ..- attr(*, "class")= chr [1:2] "element_text" "element"
##  $ plot.tag             :List of 11
##   ..$ family       : NULL
##   ..$ face         : NULL
##   ..$ colour       : NULL
##   ..$ size         :Class 'rel'  num 1.2
##   ..$ hjust        : num 0.5
##   ..$ vjust        : num 0.5
##   ..$ angle        : NULL
##   ..$ lineheight   : NULL
##   ..$ margin       : NULL
##   ..$ debug        : NULL
##   ..$ inherit.blank: logi TRUE
##   ..- attr(*, "class")= chr [1:2] "element_text" "element"
##  $ plot.tag.position    : chr "topleft"
##  $ plot.margin          :Classes 'margin', 'unit'  atomic [1:4] 5.5 5.5 5.5 5.5
##   .. ..- attr(*, "valid.unit")= int 8
##   .. ..- attr(*, "unit")= chr "pt"
##  $ strip.background     :List of 5
##   ..$ fill         : chr "white"
##   ..$ colour       : chr "black"
##   ..$ size         :Class 'rel'  num 2
##   ..$ linetype     : NULL
##   ..$ inherit.blank: logi TRUE
##   ..- attr(*, "class")= chr [1:2] "element_rect" "element"
##  $ strip.placement      : chr "inside"
##  $ strip.text           :List of 11
##   ..$ family       : NULL
##   ..$ face         : NULL
##   ..$ colour       : chr "grey10"
##   ..$ size         :Class 'rel'  num 0.8
##   ..$ hjust        : NULL
##   ..$ vjust        : NULL
##   ..$ angle        : NULL
##   ..$ lineheight   : NULL
##   ..$ margin       :Classes 'margin', 'unit'  atomic [1:4] 4.4 4.4 4.4 4.4
##   .. .. ..- attr(*, "valid.unit")= int 8
##   .. .. ..- attr(*, "unit")= chr "pt"
##   ..$ debug        : NULL
##   ..$ inherit.blank: logi TRUE
##   ..- attr(*, "class")= chr [1:2] "element_text" "element"
##  $ strip.text.x         : NULL
##  $ strip.text.y         :List of 11
##   ..$ family       : NULL
##   ..$ face         : NULL
##   ..$ colour       : NULL
##   ..$ size         : NULL
##   ..$ hjust        : NULL
##   ..$ vjust        : NULL
##   ..$ angle        : num -90
##   ..$ lineheight   : NULL
##   ..$ margin       : NULL
##   ..$ debug        : NULL
##   ..$ inherit.blank: logi TRUE
##   ..- attr(*, "class")= chr [1:2] "element_text" "element"
##  $ strip.switch.pad.grid:Class 'unit'  atomic [1:1] 2.75
##   .. ..- attr(*, "valid.unit")= int 8
##   .. ..- attr(*, "unit")= chr "pt"
##  $ strip.switch.pad.wrap:Class 'unit'  atomic [1:1] 2.75
##   .. ..- attr(*, "valid.unit")= int 8
##   .. ..- attr(*, "unit")= chr "pt"
##  $ panel.grid.major     : list()
##   ..- attr(*, "class")= chr [1:2] "element_blank" "element"
##  - attr(*, "class")= chr [1:2] "theme" "gg"
##  - attr(*, "complete")= logi TRUE
##  - attr(*, "validate")= logi TRUE
\end{verbatim}

\includegraphics{Data_Exploration_Thesis_files/figure-latex/pressure-2.pdf}
\includegraphics{Data_Exploration_Thesis_files/figure-latex/pressure-3.pdf}
\includegraphics{Data_Exploration_Thesis_files/figure-latex/pressure-4.pdf}
\includegraphics{Data_Exploration_Thesis_files/figure-latex/pressure-5.pdf}
\includegraphics{Data_Exploration_Thesis_files/figure-latex/pressure-6.pdf}
\includegraphics{Data_Exploration_Thesis_files/figure-latex/pressure-7.pdf}
\includegraphics{Data_Exploration_Thesis_files/figure-latex/pressure-8.pdf}

\subsection{Sampling location}\label{sampling-location}

TO DO: map met locaties aangeduid op satellietafbeelding

VA\_PG\_LO\_PA: Veurne Ambacht canal pumping station left bank eel
ladder VA\_PG\_RO\_PA: Veurne Ambacht canal pumping station right bank
eel ladder VA\_PG\_LO\_SU: Veurne Ambacht canal pumping station left
bank substrates VA\_PG\_RO\_SU: Veurne Ambacht canal pumping station
right bank substrates VA\_MI\_LO\_SU: Veurne Ambacht canal middle left
bank substrates VA\_MI\_RO\_SU: Veurne Ambacht canal middle right bank
substrates VA\_SC\_LO\_SU: Veurne Ambacht canal sluicecomplex Ganzepoot
left bank substrates VA\_SC\_RO\_SU: Veurne Ambacht canal sluicecomplex
Ganzepoot right bank substrates VA\_PG\_KR\_NA: Veurne Ambacht canal
pumpingstation liftnets night GA\_SU: Ganzepoot substrates

\subsection{Piecharts}\label{piecharts}

\url{https://www.zevross.com/blog/2017/06/19/tips-and-tricks-for-working-with-images-and-figures-in-r-markdown-documents/}

\begin{Shaded}
\begin{Highlighting}[]
\NormalTok{colnames <-}\StringTok{ }\KeywordTok{c}\NormalTok{(}\StringTok{"Microplastics.Contaminatie.vezels..katoen.."}\NormalTok{, }\StringTok{"Cladocera.sp."}\NormalTok{, }\StringTok{"Annelida"}\NormalTok{, }\StringTok{"Polychaete"}\NormalTok{, }\StringTok{"Isopoda"}\NormalTok{, }\StringTok{"Diptera.sp."}\NormalTok{, }\StringTok{"Chironomida.sp."}\NormalTok{, }\StringTok{"Crustaceae.sp."}\NormalTok{, }\StringTok{"Cycloida.sp."}\NormalTok{, }\StringTok{"Unidentified.sp."}\NormalTok{, }\StringTok{'Amphipoda.sp.'}\NormalTok{, }\StringTok{'Calanoida.sp.'}\NormalTok{, }\StringTok{"Copepoda.sp."}\NormalTok{, }\StringTok{"Harpacticoid.Copepod"}\NormalTok{, }\StringTok{"Hexapode.sp...."}\NormalTok{, }\StringTok{"keverlarve"}\NormalTok{, }\StringTok{"Pennate.diatomee"}\NormalTok{,}\StringTok{"Plantae.sp."}\NormalTok{)}

\ControlFlowTok{for}\NormalTok{ (i }\ControlFlowTok{in} \DecValTok{1}\OperatorTok{:}\KeywordTok{length}\NormalTok{(colnames)) \{}
\NormalTok{  bp <-}\StringTok{ }\KeywordTok{ggplot}\NormalTok{(maaginhoud2, }\KeywordTok{aes_string}\NormalTok{(}\DataTypeTok{x=}\StringTok{"''"}\NormalTok{, }\DataTypeTok{y =}\NormalTok{ colnames[i], }\DataTypeTok{fill =} \StringTok{"Methode"}\NormalTok{)) }\OperatorTok{+}
\StringTok{    }\KeywordTok{geom_bar}\NormalTok{(}\DataTypeTok{width =} \DecValTok{1}\NormalTok{, }\DataTypeTok{stat=}\StringTok{"identity"}\NormalTok{)}
\NormalTok{  pie <-}\StringTok{ }\NormalTok{bp }\OperatorTok{+}\StringTok{ }\KeywordTok{coord_polar}\NormalTok{(}\StringTok{"y"}\NormalTok{, }\DataTypeTok{start=} \DecValTok{0}\NormalTok{)}
  \KeywordTok{print}\NormalTok{(pie)}
\NormalTok{\}}
\end{Highlighting}
\end{Shaded}

\begin{verbatim}
## Warning: Removed 36 rows containing missing values (position_stack).
\end{verbatim}

\includegraphics{Data_Exploration_Thesis_files/figure-latex/unnamed-chunk-2-1.pdf}

\begin{verbatim}
## Warning: Removed 36 rows containing missing values (position_stack).
\end{verbatim}

\includegraphics{Data_Exploration_Thesis_files/figure-latex/unnamed-chunk-2-2.pdf}

\begin{verbatim}
## Warning: Removed 36 rows containing missing values (position_stack).
\end{verbatim}

\includegraphics{Data_Exploration_Thesis_files/figure-latex/unnamed-chunk-2-3.pdf}

\begin{verbatim}
## Warning: Removed 36 rows containing missing values (position_stack).
\end{verbatim}

\includegraphics{Data_Exploration_Thesis_files/figure-latex/unnamed-chunk-2-4.pdf}

\begin{verbatim}
## Warning: Removed 36 rows containing missing values (position_stack).
\end{verbatim}

\includegraphics{Data_Exploration_Thesis_files/figure-latex/unnamed-chunk-2-5.pdf}

\begin{verbatim}
## Warning: Removed 36 rows containing missing values (position_stack).
\end{verbatim}

\includegraphics{Data_Exploration_Thesis_files/figure-latex/unnamed-chunk-2-6.pdf}

\begin{verbatim}
## Warning: Removed 36 rows containing missing values (position_stack).
\end{verbatim}

\includegraphics{Data_Exploration_Thesis_files/figure-latex/unnamed-chunk-2-7.pdf}

\begin{verbatim}
## Warning: Removed 36 rows containing missing values (position_stack).
\end{verbatim}

\includegraphics{Data_Exploration_Thesis_files/figure-latex/unnamed-chunk-2-8.pdf}

\begin{verbatim}
## Warning: Removed 36 rows containing missing values (position_stack).
\end{verbatim}

\includegraphics{Data_Exploration_Thesis_files/figure-latex/unnamed-chunk-2-9.pdf}

\begin{verbatim}
## Warning: Removed 36 rows containing missing values (position_stack).
\end{verbatim}

\includegraphics{Data_Exploration_Thesis_files/figure-latex/unnamed-chunk-2-10.pdf}

\begin{verbatim}
## Warning: Removed 36 rows containing missing values (position_stack).
\end{verbatim}

\includegraphics{Data_Exploration_Thesis_files/figure-latex/unnamed-chunk-2-11.pdf}

\begin{verbatim}
## Warning: Removed 36 rows containing missing values (position_stack).
\end{verbatim}

\includegraphics{Data_Exploration_Thesis_files/figure-latex/unnamed-chunk-2-12.pdf}

\begin{verbatim}
## Warning: Removed 36 rows containing missing values (position_stack).
\end{verbatim}

\includegraphics{Data_Exploration_Thesis_files/figure-latex/unnamed-chunk-2-13.pdf}

\begin{verbatim}
## Warning: Removed 36 rows containing missing values (position_stack).
\end{verbatim}

\includegraphics{Data_Exploration_Thesis_files/figure-latex/unnamed-chunk-2-14.pdf}

\begin{verbatim}
## Warning: Removed 36 rows containing missing values (position_stack).
\end{verbatim}

\includegraphics{Data_Exploration_Thesis_files/figure-latex/unnamed-chunk-2-15.pdf}

\begin{verbatim}
## Warning: Removed 36 rows containing missing values (position_stack).
\end{verbatim}

\includegraphics{Data_Exploration_Thesis_files/figure-latex/unnamed-chunk-2-16.pdf}

\begin{verbatim}
## Warning: Removed 36 rows containing missing values (position_stack).
\end{verbatim}

\includegraphics{Data_Exploration_Thesis_files/figure-latex/unnamed-chunk-2-17.pdf}

\begin{verbatim}
## Warning: Removed 36 rows containing missing values (position_stack).
\end{verbatim}

\includegraphics{Data_Exploration_Thesis_files/figure-latex/unnamed-chunk-2-18.pdf}

\begin{Shaded}
\begin{Highlighting}[]
\ControlFlowTok{for}\NormalTok{ (i }\ControlFlowTok{in} \DecValTok{1}\OperatorTok{:}\KeywordTok{length}\NormalTok{(colnames)) \{}
\NormalTok{  bp <-}\StringTok{ }\KeywordTok{ggplot}\NormalTok{(maaginhoud2, }\KeywordTok{aes_string}\NormalTok{(}\DataTypeTok{x=}\StringTok{"''"}\NormalTok{, }\DataTypeTok{y =}\NormalTok{ colnames[i], }\DataTypeTok{fill =} \StringTok{"Locatie_A"}\NormalTok{)) }\OperatorTok{+}
\StringTok{    }\KeywordTok{geom_bar}\NormalTok{(}\DataTypeTok{width =} \DecValTok{1}\NormalTok{, }\DataTypeTok{stat=}\StringTok{"identity"}\NormalTok{)}
\NormalTok{  pie <-}\StringTok{ }\NormalTok{bp }\OperatorTok{+}\StringTok{ }\KeywordTok{coord_polar}\NormalTok{(}\StringTok{"y"}\NormalTok{, }\DataTypeTok{start=} \DecValTok{0}\NormalTok{)}
  \KeywordTok{print}\NormalTok{(pie)}
\NormalTok{\}}
\end{Highlighting}
\end{Shaded}

\begin{verbatim}
## Warning: Removed 36 rows containing missing values (position_stack).
\end{verbatim}

\includegraphics{Data_Exploration_Thesis_files/figure-latex/unnamed-chunk-2-19.pdf}

\begin{verbatim}
## Warning: Removed 36 rows containing missing values (position_stack).
\end{verbatim}

\includegraphics{Data_Exploration_Thesis_files/figure-latex/unnamed-chunk-2-20.pdf}

\begin{verbatim}
## Warning: Removed 36 rows containing missing values (position_stack).
\end{verbatim}

\includegraphics{Data_Exploration_Thesis_files/figure-latex/unnamed-chunk-2-21.pdf}

\begin{verbatim}
## Warning: Removed 36 rows containing missing values (position_stack).
\end{verbatim}

\includegraphics{Data_Exploration_Thesis_files/figure-latex/unnamed-chunk-2-22.pdf}

\begin{verbatim}
## Warning: Removed 36 rows containing missing values (position_stack).
\end{verbatim}

\includegraphics{Data_Exploration_Thesis_files/figure-latex/unnamed-chunk-2-23.pdf}

\begin{verbatim}
## Warning: Removed 36 rows containing missing values (position_stack).
\end{verbatim}

\includegraphics{Data_Exploration_Thesis_files/figure-latex/unnamed-chunk-2-24.pdf}

\begin{verbatim}
## Warning: Removed 36 rows containing missing values (position_stack).
\end{verbatim}

\includegraphics{Data_Exploration_Thesis_files/figure-latex/unnamed-chunk-2-25.pdf}

\begin{verbatim}
## Warning: Removed 36 rows containing missing values (position_stack).
\end{verbatim}

\includegraphics{Data_Exploration_Thesis_files/figure-latex/unnamed-chunk-2-26.pdf}

\begin{verbatim}
## Warning: Removed 36 rows containing missing values (position_stack).
\end{verbatim}

\includegraphics{Data_Exploration_Thesis_files/figure-latex/unnamed-chunk-2-27.pdf}

\begin{verbatim}
## Warning: Removed 36 rows containing missing values (position_stack).
\end{verbatim}

\includegraphics{Data_Exploration_Thesis_files/figure-latex/unnamed-chunk-2-28.pdf}

\begin{verbatim}
## Warning: Removed 36 rows containing missing values (position_stack).
\end{verbatim}

\includegraphics{Data_Exploration_Thesis_files/figure-latex/unnamed-chunk-2-29.pdf}

\begin{verbatim}
## Warning: Removed 36 rows containing missing values (position_stack).
\end{verbatim}

\includegraphics{Data_Exploration_Thesis_files/figure-latex/unnamed-chunk-2-30.pdf}

\begin{verbatim}
## Warning: Removed 36 rows containing missing values (position_stack).
\end{verbatim}

\includegraphics{Data_Exploration_Thesis_files/figure-latex/unnamed-chunk-2-31.pdf}

\begin{verbatim}
## Warning: Removed 36 rows containing missing values (position_stack).
\end{verbatim}

\includegraphics{Data_Exploration_Thesis_files/figure-latex/unnamed-chunk-2-32.pdf}

\begin{verbatim}
## Warning: Removed 36 rows containing missing values (position_stack).
\end{verbatim}

\includegraphics{Data_Exploration_Thesis_files/figure-latex/unnamed-chunk-2-33.pdf}

\begin{verbatim}
## Warning: Removed 36 rows containing missing values (position_stack).
\end{verbatim}

\includegraphics{Data_Exploration_Thesis_files/figure-latex/unnamed-chunk-2-34.pdf}

\begin{verbatim}
## Warning: Removed 36 rows containing missing values (position_stack).
\end{verbatim}

\includegraphics{Data_Exploration_Thesis_files/figure-latex/unnamed-chunk-2-35.pdf}

\begin{verbatim}
## Warning: Removed 36 rows containing missing values (position_stack).
\end{verbatim}

\includegraphics{Data_Exploration_Thesis_files/figure-latex/unnamed-chunk-2-36.pdf}

But now, we want to constructs piecharts which are not absolute numbers,
but relative numbers. For example, if a glass eel ate 1 polychaete out
of the total of 50 found polychaetes, this results in a relative number
on the pie chart of 2\%

\begin{Shaded}
\begin{Highlighting}[]
\NormalTok{maaginhoud2 <-}\StringTok{ }\KeywordTok{remove_empty}\NormalTok{(maaginhoud2, }\StringTok{'rows'}\NormalTok{)}
\NormalTok{col_sum <-}\StringTok{ }\KeywordTok{colSums}\NormalTok{(maaginhoud2[}\DecValTok{7}\OperatorTok{:}\DecValTok{24}\NormalTok{], }\DataTypeTok{na.rm =} \OtherTok{TRUE}\NormalTok{)}

\NormalTok{maaginhoud3 <-}\StringTok{ }\NormalTok{maaginhoud2[}\DecValTok{1}\OperatorTok{:}\DecValTok{6}\NormalTok{]}
\NormalTok{new_column <-}\StringTok{ }\KeywordTok{list}\NormalTok{()}

\ControlFlowTok{for}\NormalTok{ (count }\ControlFlowTok{in} \DecValTok{7}\OperatorTok{:}\KeywordTok{length}\NormalTok{(maaginhoud2)) \{}
\NormalTok{  new_column <-}\StringTok{ }\NormalTok{(maaginhoud2[count] }\OperatorTok{/}\StringTok{ }\NormalTok{col_sum[count }\OperatorTok{-}\StringTok{ }\DecValTok{6}\NormalTok{])}
\NormalTok{  maaginhoud3 <-}\StringTok{ }\KeywordTok{cbind}\NormalTok{(maaginhoud3, new_column)}
\NormalTok{\}}

\ControlFlowTok{for}\NormalTok{ (i }\ControlFlowTok{in} \DecValTok{1}\OperatorTok{:}\KeywordTok{length}\NormalTok{(colnames)) \{}
\NormalTok{  bp <-}\StringTok{ }\KeywordTok{ggplot}\NormalTok{(maaginhoud3, }\KeywordTok{aes_string}\NormalTok{(}\DataTypeTok{x=}\StringTok{"''"}\NormalTok{, }\DataTypeTok{y =}\NormalTok{ colnames[i], }\DataTypeTok{fill =} \StringTok{"Methode"}\NormalTok{)) }\OperatorTok{+}
\StringTok{    }\KeywordTok{geom_bar}\NormalTok{(}\DataTypeTok{width =} \DecValTok{1}\NormalTok{, }\DataTypeTok{stat=}\StringTok{"identity"}\NormalTok{)}
\NormalTok{  pie <-}\StringTok{ }\NormalTok{bp }\OperatorTok{+}\StringTok{ }\KeywordTok{coord_polar}\NormalTok{(}\StringTok{"y"}\NormalTok{, }\DataTypeTok{start=} \DecValTok{0}\NormalTok{)}
  \KeywordTok{print}\NormalTok{(pie)}
\NormalTok{\}}
\end{Highlighting}
\end{Shaded}

\begin{verbatim}
## Warning: Removed 36 rows containing missing values (position_stack).
\end{verbatim}

\includegraphics{Data_Exploration_Thesis_files/figure-latex/unnamed-chunk-3-1.pdf}

\begin{verbatim}
## Warning: Removed 36 rows containing missing values (position_stack).
\end{verbatim}

\includegraphics{Data_Exploration_Thesis_files/figure-latex/unnamed-chunk-3-2.pdf}

\begin{verbatim}
## Warning: Removed 36 rows containing missing values (position_stack).
\end{verbatim}

\includegraphics{Data_Exploration_Thesis_files/figure-latex/unnamed-chunk-3-3.pdf}

\begin{verbatim}
## Warning: Removed 36 rows containing missing values (position_stack).
\end{verbatim}

\includegraphics{Data_Exploration_Thesis_files/figure-latex/unnamed-chunk-3-4.pdf}

\begin{verbatim}
## Warning: Removed 36 rows containing missing values (position_stack).
\end{verbatim}

\includegraphics{Data_Exploration_Thesis_files/figure-latex/unnamed-chunk-3-5.pdf}

\begin{verbatim}
## Warning: Removed 36 rows containing missing values (position_stack).
\end{verbatim}

\includegraphics{Data_Exploration_Thesis_files/figure-latex/unnamed-chunk-3-6.pdf}

\begin{verbatim}
## Warning: Removed 36 rows containing missing values (position_stack).
\end{verbatim}

\includegraphics{Data_Exploration_Thesis_files/figure-latex/unnamed-chunk-3-7.pdf}

\begin{verbatim}
## Warning: Removed 36 rows containing missing values (position_stack).
\end{verbatim}

\includegraphics{Data_Exploration_Thesis_files/figure-latex/unnamed-chunk-3-8.pdf}

\begin{verbatim}
## Warning: Removed 36 rows containing missing values (position_stack).
\end{verbatim}

\includegraphics{Data_Exploration_Thesis_files/figure-latex/unnamed-chunk-3-9.pdf}

\begin{verbatim}
## Warning: Removed 36 rows containing missing values (position_stack).
\end{verbatim}

\includegraphics{Data_Exploration_Thesis_files/figure-latex/unnamed-chunk-3-10.pdf}

\begin{verbatim}
## Warning: Removed 36 rows containing missing values (position_stack).
\end{verbatim}

\includegraphics{Data_Exploration_Thesis_files/figure-latex/unnamed-chunk-3-11.pdf}

\begin{verbatim}
## Warning: Removed 36 rows containing missing values (position_stack).
\end{verbatim}

\includegraphics{Data_Exploration_Thesis_files/figure-latex/unnamed-chunk-3-12.pdf}

\begin{verbatim}
## Warning: Removed 36 rows containing missing values (position_stack).
\end{verbatim}

\includegraphics{Data_Exploration_Thesis_files/figure-latex/unnamed-chunk-3-13.pdf}

\begin{verbatim}
## Warning: Removed 36 rows containing missing values (position_stack).
\end{verbatim}

\includegraphics{Data_Exploration_Thesis_files/figure-latex/unnamed-chunk-3-14.pdf}

\begin{verbatim}
## Warning: Removed 36 rows containing missing values (position_stack).
\end{verbatim}

\includegraphics{Data_Exploration_Thesis_files/figure-latex/unnamed-chunk-3-15.pdf}

\begin{verbatim}
## Warning: Removed 36 rows containing missing values (position_stack).
\end{verbatim}

\includegraphics{Data_Exploration_Thesis_files/figure-latex/unnamed-chunk-3-16.pdf}

\begin{verbatim}
## Warning: Removed 36 rows containing missing values (position_stack).
\end{verbatim}

\includegraphics{Data_Exploration_Thesis_files/figure-latex/unnamed-chunk-3-17.pdf}

\begin{verbatim}
## Warning: Removed 36 rows containing missing values (position_stack).
\end{verbatim}

\includegraphics{Data_Exploration_Thesis_files/figure-latex/unnamed-chunk-3-18.pdf}

\begin{Shaded}
\begin{Highlighting}[]
\ControlFlowTok{for}\NormalTok{ (i }\ControlFlowTok{in} \DecValTok{1}\OperatorTok{:}\KeywordTok{length}\NormalTok{(colnames)) \{}
\NormalTok{  bp <-}\StringTok{ }\KeywordTok{ggplot}\NormalTok{(maaginhoud3, }\KeywordTok{aes_string}\NormalTok{(}\DataTypeTok{x=}\StringTok{"''"}\NormalTok{, }\DataTypeTok{y =}\NormalTok{ colnames[i], }\DataTypeTok{fill =} \StringTok{"Locatie_A"}\NormalTok{)) }\OperatorTok{+}
\StringTok{    }\KeywordTok{geom_bar}\NormalTok{(}\DataTypeTok{width =} \DecValTok{1}\NormalTok{, }\DataTypeTok{stat=}\StringTok{"identity"}\NormalTok{)}
\NormalTok{  pie <-}\StringTok{ }\NormalTok{bp }\OperatorTok{+}\StringTok{ }\KeywordTok{coord_polar}\NormalTok{(}\StringTok{"y"}\NormalTok{, }\DataTypeTok{start=} \DecValTok{0}\NormalTok{)}
  \KeywordTok{print}\NormalTok{(pie)}
\NormalTok{\}}
\end{Highlighting}
\end{Shaded}

\begin{verbatim}
## Warning: Removed 36 rows containing missing values (position_stack).
\end{verbatim}

\includegraphics{Data_Exploration_Thesis_files/figure-latex/unnamed-chunk-3-19.pdf}

\begin{verbatim}
## Warning: Removed 36 rows containing missing values (position_stack).
\end{verbatim}

\includegraphics{Data_Exploration_Thesis_files/figure-latex/unnamed-chunk-3-20.pdf}

\begin{verbatim}
## Warning: Removed 36 rows containing missing values (position_stack).
\end{verbatim}

\includegraphics{Data_Exploration_Thesis_files/figure-latex/unnamed-chunk-3-21.pdf}

\begin{verbatim}
## Warning: Removed 36 rows containing missing values (position_stack).
\end{verbatim}

\includegraphics{Data_Exploration_Thesis_files/figure-latex/unnamed-chunk-3-22.pdf}

\begin{verbatim}
## Warning: Removed 36 rows containing missing values (position_stack).
\end{verbatim}

\includegraphics{Data_Exploration_Thesis_files/figure-latex/unnamed-chunk-3-23.pdf}

\begin{verbatim}
## Warning: Removed 36 rows containing missing values (position_stack).
\end{verbatim}

\includegraphics{Data_Exploration_Thesis_files/figure-latex/unnamed-chunk-3-24.pdf}

\begin{verbatim}
## Warning: Removed 36 rows containing missing values (position_stack).
\end{verbatim}

\includegraphics{Data_Exploration_Thesis_files/figure-latex/unnamed-chunk-3-25.pdf}

\begin{verbatim}
## Warning: Removed 36 rows containing missing values (position_stack).
\end{verbatim}

\includegraphics{Data_Exploration_Thesis_files/figure-latex/unnamed-chunk-3-26.pdf}

\begin{verbatim}
## Warning: Removed 36 rows containing missing values (position_stack).
\end{verbatim}

\includegraphics{Data_Exploration_Thesis_files/figure-latex/unnamed-chunk-3-27.pdf}

\begin{verbatim}
## Warning: Removed 36 rows containing missing values (position_stack).
\end{verbatim}

\includegraphics{Data_Exploration_Thesis_files/figure-latex/unnamed-chunk-3-28.pdf}

\begin{verbatim}
## Warning: Removed 36 rows containing missing values (position_stack).
\end{verbatim}

\includegraphics{Data_Exploration_Thesis_files/figure-latex/unnamed-chunk-3-29.pdf}

\begin{verbatim}
## Warning: Removed 36 rows containing missing values (position_stack).
\end{verbatim}

\includegraphics{Data_Exploration_Thesis_files/figure-latex/unnamed-chunk-3-30.pdf}

\begin{verbatim}
## Warning: Removed 36 rows containing missing values (position_stack).
\end{verbatim}

\includegraphics{Data_Exploration_Thesis_files/figure-latex/unnamed-chunk-3-31.pdf}

\begin{verbatim}
## Warning: Removed 36 rows containing missing values (position_stack).
\end{verbatim}

\includegraphics{Data_Exploration_Thesis_files/figure-latex/unnamed-chunk-3-32.pdf}

\begin{verbatim}
## Warning: Removed 36 rows containing missing values (position_stack).
\end{verbatim}

\includegraphics{Data_Exploration_Thesis_files/figure-latex/unnamed-chunk-3-33.pdf}

\begin{verbatim}
## Warning: Removed 36 rows containing missing values (position_stack).
\end{verbatim}

\includegraphics{Data_Exploration_Thesis_files/figure-latex/unnamed-chunk-3-34.pdf}

\begin{verbatim}
## Warning: Removed 36 rows containing missing values (position_stack).
\end{verbatim}

\includegraphics{Data_Exploration_Thesis_files/figure-latex/unnamed-chunk-3-35.pdf}

\begin{verbatim}
## Warning: Removed 36 rows containing missing values (position_stack).
\end{verbatim}

\includegraphics{Data_Exploration_Thesis_files/figure-latex/unnamed-chunk-3-36.pdf}

Now we want to compose relative piecharts divided according to
pigmentation stage

\begin{Shaded}
\begin{Highlighting}[]
\NormalTok{data_inbo_pigmentation <-}\StringTok{ }\NormalTok{data_inbo[,}\KeywordTok{c}\NormalTok{(}\StringTok{"nr"}\NormalTok{,}\StringTok{"pigmentation.stage"}\NormalTok{)]}
\NormalTok{maaginhoud4 <-}\StringTok{ }\KeywordTok{left_join}\NormalTok{(maaginhoud2, data_inbo_pigmentation, }\DataTypeTok{by=}\StringTok{'nr'}\NormalTok{)}

\ControlFlowTok{for}\NormalTok{ (i }\ControlFlowTok{in} \DecValTok{1}\OperatorTok{:}\KeywordTok{length}\NormalTok{(colnames)) \{}
\NormalTok{  bp <-}\StringTok{ }\KeywordTok{ggplot}\NormalTok{(maaginhoud4, }\KeywordTok{aes_string}\NormalTok{(}\DataTypeTok{x=}\StringTok{"''"}\NormalTok{, }\DataTypeTok{y =}\NormalTok{ colnames[i], }\DataTypeTok{fill =} \StringTok{"pigmentation.stage"}\NormalTok{)) }\OperatorTok{+}
\StringTok{    }\KeywordTok{geom_bar}\NormalTok{(}\DataTypeTok{width =} \DecValTok{1}\NormalTok{, }\DataTypeTok{stat=}\StringTok{"identity"}\NormalTok{)}
\NormalTok{  pie <-}\StringTok{ }\NormalTok{bp }\OperatorTok{+}\StringTok{ }\KeywordTok{coord_polar}\NormalTok{(}\StringTok{"y"}\NormalTok{, }\DataTypeTok{start=} \DecValTok{0}\NormalTok{)}
  \KeywordTok{print}\NormalTok{(pie)}
\NormalTok{\}}
\end{Highlighting}
\end{Shaded}

\begin{verbatim}
## Warning: Removed 36 rows containing missing values (position_stack).
\end{verbatim}

\includegraphics{Data_Exploration_Thesis_files/figure-latex/unnamed-chunk-4-1.pdf}

\begin{verbatim}
## Warning: Removed 36 rows containing missing values (position_stack).
\end{verbatim}

\includegraphics{Data_Exploration_Thesis_files/figure-latex/unnamed-chunk-4-2.pdf}

\begin{verbatim}
## Warning: Removed 36 rows containing missing values (position_stack).
\end{verbatim}

\includegraphics{Data_Exploration_Thesis_files/figure-latex/unnamed-chunk-4-3.pdf}

\begin{verbatim}
## Warning: Removed 36 rows containing missing values (position_stack).
\end{verbatim}

\includegraphics{Data_Exploration_Thesis_files/figure-latex/unnamed-chunk-4-4.pdf}

\begin{verbatim}
## Warning: Removed 36 rows containing missing values (position_stack).
\end{verbatim}

\includegraphics{Data_Exploration_Thesis_files/figure-latex/unnamed-chunk-4-5.pdf}

\begin{verbatim}
## Warning: Removed 36 rows containing missing values (position_stack).
\end{verbatim}

\includegraphics{Data_Exploration_Thesis_files/figure-latex/unnamed-chunk-4-6.pdf}

\begin{verbatim}
## Warning: Removed 36 rows containing missing values (position_stack).
\end{verbatim}

\includegraphics{Data_Exploration_Thesis_files/figure-latex/unnamed-chunk-4-7.pdf}

\begin{verbatim}
## Warning: Removed 36 rows containing missing values (position_stack).
\end{verbatim}

\includegraphics{Data_Exploration_Thesis_files/figure-latex/unnamed-chunk-4-8.pdf}

\begin{verbatim}
## Warning: Removed 36 rows containing missing values (position_stack).
\end{verbatim}

\includegraphics{Data_Exploration_Thesis_files/figure-latex/unnamed-chunk-4-9.pdf}

\begin{verbatim}
## Warning: Removed 36 rows containing missing values (position_stack).
\end{verbatim}

\includegraphics{Data_Exploration_Thesis_files/figure-latex/unnamed-chunk-4-10.pdf}

\begin{verbatim}
## Warning: Removed 36 rows containing missing values (position_stack).
\end{verbatim}

\includegraphics{Data_Exploration_Thesis_files/figure-latex/unnamed-chunk-4-11.pdf}

\begin{verbatim}
## Warning: Removed 36 rows containing missing values (position_stack).
\end{verbatim}

\includegraphics{Data_Exploration_Thesis_files/figure-latex/unnamed-chunk-4-12.pdf}

\begin{verbatim}
## Warning: Removed 36 rows containing missing values (position_stack).
\end{verbatim}

\includegraphics{Data_Exploration_Thesis_files/figure-latex/unnamed-chunk-4-13.pdf}

\begin{verbatim}
## Warning: Removed 36 rows containing missing values (position_stack).
\end{verbatim}

\includegraphics{Data_Exploration_Thesis_files/figure-latex/unnamed-chunk-4-14.pdf}

\begin{verbatim}
## Warning: Removed 36 rows containing missing values (position_stack).
\end{verbatim}

\includegraphics{Data_Exploration_Thesis_files/figure-latex/unnamed-chunk-4-15.pdf}

\begin{verbatim}
## Warning: Removed 36 rows containing missing values (position_stack).
\end{verbatim}

\includegraphics{Data_Exploration_Thesis_files/figure-latex/unnamed-chunk-4-16.pdf}

\begin{verbatim}
## Warning: Removed 36 rows containing missing values (position_stack).
\end{verbatim}

\includegraphics{Data_Exploration_Thesis_files/figure-latex/unnamed-chunk-4-17.pdf}

\begin{verbatim}
## Warning: Removed 36 rows containing missing values (position_stack).
\end{verbatim}

\includegraphics{Data_Exploration_Thesis_files/figure-latex/unnamed-chunk-4-18.pdf}

Now we want to compare what the relative percentage caught per method \&
location is To do so, we need a transposed data table. This needs some
more work

\subsection{Bargraphs}\label{bargraphs}

Hier wordt weergegeven welke organismen op welke tijd van het jaar in
verschillende pigmentatiestadia van de glasaal teruggevonden werden.

\begin{Shaded}
\begin{Highlighting}[]
\ControlFlowTok{for}\NormalTok{ (i }\ControlFlowTok{in} \DecValTok{1}\OperatorTok{:}\KeywordTok{length}\NormalTok{(colnames)) \{}
\NormalTok{  bp <-}\StringTok{ }\KeywordTok{ggplot}\NormalTok{(maaginhoud4, }\KeywordTok{aes_string}\NormalTok{(}\DataTypeTok{x=}\KeywordTok{week}\NormalTok{(}\KeywordTok{as.Date}\NormalTok{(maaginhoud4}\OperatorTok{$}\NormalTok{datum)), }\DataTypeTok{y =}\NormalTok{ colnames[i], }\DataTypeTok{fill =} \StringTok{"pigmentation.stage"}\NormalTok{)) }\OperatorTok{+}
\StringTok{    }\KeywordTok{geom_bar}\NormalTok{(}\DataTypeTok{width =} \DecValTok{1}\NormalTok{, }\DataTypeTok{stat=}\StringTok{"identity"}\NormalTok{) }\OperatorTok{+}
\StringTok{    }\KeywordTok{xlab}\NormalTok{(}\DataTypeTok{label =} \StringTok{"Week of 2017"}\NormalTok{)}
  \KeywordTok{print}\NormalTok{(bp)}
\NormalTok{\}}
\end{Highlighting}
\end{Shaded}

\begin{verbatim}
## Warning: Removed 36 rows containing missing values (position_stack).
\end{verbatim}

\includegraphics{Data_Exploration_Thesis_files/figure-latex/unnamed-chunk-6-1.pdf}

\begin{verbatim}
## Warning: Removed 36 rows containing missing values (position_stack).
\end{verbatim}

\includegraphics{Data_Exploration_Thesis_files/figure-latex/unnamed-chunk-6-2.pdf}

\begin{verbatim}
## Warning: Removed 36 rows containing missing values (position_stack).
\end{verbatim}

\includegraphics{Data_Exploration_Thesis_files/figure-latex/unnamed-chunk-6-3.pdf}

\begin{verbatim}
## Warning: Removed 36 rows containing missing values (position_stack).
\end{verbatim}

\includegraphics{Data_Exploration_Thesis_files/figure-latex/unnamed-chunk-6-4.pdf}

\begin{verbatim}
## Warning: Removed 36 rows containing missing values (position_stack).
\end{verbatim}

\includegraphics{Data_Exploration_Thesis_files/figure-latex/unnamed-chunk-6-5.pdf}

\begin{verbatim}
## Warning: Removed 36 rows containing missing values (position_stack).
\end{verbatim}

\includegraphics{Data_Exploration_Thesis_files/figure-latex/unnamed-chunk-6-6.pdf}

\begin{verbatim}
## Warning: Removed 36 rows containing missing values (position_stack).
\end{verbatim}

\includegraphics{Data_Exploration_Thesis_files/figure-latex/unnamed-chunk-6-7.pdf}

\begin{verbatim}
## Warning: Removed 36 rows containing missing values (position_stack).
\end{verbatim}

\includegraphics{Data_Exploration_Thesis_files/figure-latex/unnamed-chunk-6-8.pdf}

\begin{verbatim}
## Warning: Removed 36 rows containing missing values (position_stack).
\end{verbatim}

\includegraphics{Data_Exploration_Thesis_files/figure-latex/unnamed-chunk-6-9.pdf}

\begin{verbatim}
## Warning: Removed 36 rows containing missing values (position_stack).
\end{verbatim}

\includegraphics{Data_Exploration_Thesis_files/figure-latex/unnamed-chunk-6-10.pdf}

\begin{verbatim}
## Warning: Removed 36 rows containing missing values (position_stack).
\end{verbatim}

\includegraphics{Data_Exploration_Thesis_files/figure-latex/unnamed-chunk-6-11.pdf}

\begin{verbatim}
## Warning: Removed 36 rows containing missing values (position_stack).
\end{verbatim}

\includegraphics{Data_Exploration_Thesis_files/figure-latex/unnamed-chunk-6-12.pdf}

\begin{verbatim}
## Warning: Removed 36 rows containing missing values (position_stack).
\end{verbatim}

\includegraphics{Data_Exploration_Thesis_files/figure-latex/unnamed-chunk-6-13.pdf}

\begin{verbatim}
## Warning: Removed 36 rows containing missing values (position_stack).
\end{verbatim}

\includegraphics{Data_Exploration_Thesis_files/figure-latex/unnamed-chunk-6-14.pdf}

\begin{verbatim}
## Warning: Removed 36 rows containing missing values (position_stack).
\end{verbatim}

\includegraphics{Data_Exploration_Thesis_files/figure-latex/unnamed-chunk-6-15.pdf}

\begin{verbatim}
## Warning: Removed 36 rows containing missing values (position_stack).
\end{verbatim}

\includegraphics{Data_Exploration_Thesis_files/figure-latex/unnamed-chunk-6-16.pdf}

\begin{verbatim}
## Warning: Removed 36 rows containing missing values (position_stack).
\end{verbatim}

\includegraphics{Data_Exploration_Thesis_files/figure-latex/unnamed-chunk-6-17.pdf}

\begin{verbatim}
## Warning: Removed 36 rows containing missing values (position_stack).
\end{verbatim}

\includegraphics{Data_Exploration_Thesis_files/figure-latex/unnamed-chunk-6-18.pdf}

\section{Environmental samples}\label{environmental-samples}

Een ideetje om de Forage ratio te berekenen. Forage ratio:

\[ \frac{percent~weight~of~food~item~i~in~stomach~of~glass~eel~w}{percent~weight~of~food~item~i~in~the~environment}~~~~ x ~~ 100  \]
Hier bekijken we eens de dataset waarin het aantal individuen per liter
werd bekeken in 2016, maar om de bovenstaande ratio te berekenen zou
toch een set van het jaar 2017 nodig zijn. Is die er?

\begin{Shaded}
\begin{Highlighting}[]
\NormalTok{environment2016_ind <-}\StringTok{ }\KeywordTok{as.tibble}\NormalTok{(}\KeywordTok{read.csv}\NormalTok{(}\StringTok{"C:/Users/user/Desktop/Biologie/Master/Thesis/Thesis/data/raw/Individuals per liter 2016.csv"}\NormalTok{, }\DataTypeTok{sep =} \StringTok{";"}\NormalTok{, }\DataTypeTok{header =} \OtherTok{TRUE}\NormalTok{, }\DataTypeTok{dec =} \StringTok{","}\NormalTok{))}

\NormalTok{env_cyclo <-}\StringTok{ }\KeywordTok{ggplot}\NormalTok{(environment2016_ind, }\KeywordTok{aes}\NormalTok{(}\KeywordTok{week}\NormalTok{(}\KeywordTok{as.Date}\NormalTok{(Datum)), Cyclopoida)) }\OperatorTok{+}
\StringTok{  }\KeywordTok{geom_point}\NormalTok{() }\OperatorTok{+}
\StringTok{  }\KeywordTok{ylab}\NormalTok{(}\StringTok{"Cyclopoida ind/L 2016"}\NormalTok{)}
\NormalTok{env_cyclo}
\end{Highlighting}
\end{Shaded}

\includegraphics{Data_Exploration_Thesis_files/figure-latex/unnamed-chunk-7-1.pdf}

\begin{Shaded}
\begin{Highlighting}[]
\NormalTok{env_alona_a <-}\StringTok{ }\KeywordTok{ggplot}\NormalTok{(environment2016_ind, }\KeywordTok{aes}\NormalTok{(}\KeywordTok{week}\NormalTok{(Datum), Alona.affinis)) }\OperatorTok{+}
\StringTok{  }\KeywordTok{geom_point}\NormalTok{() }\OperatorTok{+}
\StringTok{  }\KeywordTok{ylab}\NormalTok{(}\StringTok{"Alona affinis ind/L 2016"}\NormalTok{)}
\NormalTok{env_alona_a}
\end{Highlighting}
\end{Shaded}

\includegraphics{Data_Exploration_Thesis_files/figure-latex/unnamed-chunk-7-2.pdf}

\begin{Shaded}
\begin{Highlighting}[]
\NormalTok{env_alona_r <-}\StringTok{ }\KeywordTok{ggplot}\NormalTok{(environment2016_ind, }\KeywordTok{aes}\NormalTok{(}\KeywordTok{week}\NormalTok{(Datum), Alona.rectangula)) }\OperatorTok{+}
\StringTok{  }\KeywordTok{geom_point}\NormalTok{() }\OperatorTok{+}
\StringTok{  }\KeywordTok{ylab}\NormalTok{(}\StringTok{"Alona rectangula ind/L 2016"}\NormalTok{)}
\NormalTok{env_alona_r}
\end{Highlighting}
\end{Shaded}

\includegraphics{Data_Exploration_Thesis_files/figure-latex/unnamed-chunk-7-3.pdf}

\begin{Shaded}
\begin{Highlighting}[]
\NormalTok{env_chydorus <-}\StringTok{ }\KeywordTok{ggplot}\NormalTok{(environment2016_ind, }\KeywordTok{aes}\NormalTok{(}\KeywordTok{week}\NormalTok{(Datum), Chydorus.sphaericus)) }\OperatorTok{+}
\StringTok{  }\KeywordTok{geom_point}\NormalTok{() }\OperatorTok{+}
\StringTok{  }\KeywordTok{ylab}\NormalTok{(}\StringTok{"Chydorus sphaericus ind/L 2016"}\NormalTok{)}
\NormalTok{env_chydorus}
\end{Highlighting}
\end{Shaded}

\includegraphics{Data_Exploration_Thesis_files/figure-latex/unnamed-chunk-7-4.pdf}

\begin{Shaded}
\begin{Highlighting}[]
\NormalTok{env_daphnia <-}\StringTok{ }\KeywordTok{ggplot}\NormalTok{(environment2016_ind, }\KeywordTok{aes}\NormalTok{(}\KeywordTok{week}\NormalTok{(Datum), Daphnia.galeata)) }\OperatorTok{+}
\StringTok{  }\KeywordTok{geom_point}\NormalTok{() }\OperatorTok{+}
\StringTok{  }\KeywordTok{ylab}\NormalTok{(}\StringTok{"Daphnia galeata ind/L 2016"}\NormalTok{)}
\NormalTok{env_daphnia }
\end{Highlighting}
\end{Shaded}

\includegraphics{Data_Exploration_Thesis_files/figure-latex/unnamed-chunk-7-5.pdf}

\begin{Shaded}
\begin{Highlighting}[]
\NormalTok{env_pleuroxus <-}\StringTok{ }\KeywordTok{ggplot}\NormalTok{(environment2016_ind, }\KeywordTok{aes}\NormalTok{(}\KeywordTok{week}\NormalTok{(Datum), Pleuroxus.aduncus)) }\OperatorTok{+}
\StringTok{  }\KeywordTok{geom_point}\NormalTok{() }\OperatorTok{+}
\StringTok{  }\KeywordTok{ylab}\NormalTok{(}\StringTok{"Pleuroxus aduncus ind/L 2016"}\NormalTok{)}
\NormalTok{env_pleuroxus}
\end{Highlighting}
\end{Shaded}

\includegraphics{Data_Exploration_Thesis_files/figure-latex/unnamed-chunk-7-6.pdf}

\section{Indices}\label{indices}

Here, we calculate some indices from the dataset and visually explore
these.

\subsection{Stomach content analysis
indices}\label{stomach-content-analysis-indices}

\subsubsection{Condition Factor}\label{condition-factor}

First, the condition factor is analysed. Condition factor is already a
variable in the dataset, calculated by the INBO.

Vraag voor jeroen: hoe werd deze conditiefactor juist berekend?

\begin{Shaded}
\begin{Highlighting}[]
\NormalTok{plot_condfact_method <-}\StringTok{ }\KeywordTok{ggplot}\NormalTok{(data_inbo, }\KeywordTok{aes}\NormalTok{(}\KeywordTok{week}\NormalTok{(Date), condition.factor)) }\OperatorTok{+}
\StringTok{  }\KeywordTok{geom_point}\NormalTok{() }\OperatorTok{+}
\StringTok{  }\KeywordTok{ggtitle}\NormalTok{(}\StringTok{"Condition factor of glass eel vs week of catchment"}\NormalTok{) }\OperatorTok{+}
\StringTok{  }\KeywordTok{facet_grid}\NormalTok{(.}\OperatorTok{~}\NormalTok{Catchment.method) }\OperatorTok{+}
\StringTok{  }\KeywordTok{xlab}\NormalTok{(}\StringTok{"Conditiefactor"}\NormalTok{) }\OperatorTok{+}
\StringTok{  }\KeywordTok{ylab}\NormalTok{(}\StringTok{"Week van het jaar 2017"}\NormalTok{)}
\NormalTok{plot_condfact_method}
\end{Highlighting}
\end{Shaded}

\includegraphics{Data_Exploration_Thesis_files/figure-latex/unnamed-chunk-8-1.pdf}

\begin{Shaded}
\begin{Highlighting}[]
\NormalTok{plot_condfact_location <-}\StringTok{ }\KeywordTok{ggplot}\NormalTok{(data_inbo, }\KeywordTok{aes}\NormalTok{(}\KeywordTok{week}\NormalTok{(Date), condition.factor)) }\OperatorTok{+}
\StringTok{  }\KeywordTok{geom_point}\NormalTok{() }\OperatorTok{+}
\StringTok{  }\KeywordTok{ggtitle}\NormalTok{(}\StringTok{"Condition factor of glass eel vs week of catchment"}\NormalTok{) }\OperatorTok{+}
\StringTok{  }\KeywordTok{facet_grid}\NormalTok{(.}\OperatorTok{~}\NormalTok{Location)}
\NormalTok{plot_condfact_location}
\end{Highlighting}
\end{Shaded}

\includegraphics{Data_Exploration_Thesis_files/figure-latex/unnamed-chunk-8-2.pdf}

\begin{Shaded}
\begin{Highlighting}[]
\NormalTok{plot_condfact_pigm <-}\StringTok{ }\KeywordTok{ggplot}\NormalTok{(data_inbo, }\KeywordTok{aes}\NormalTok{(pigmentation.stage, condition.factor)) }\OperatorTok{+}
\StringTok{  }\KeywordTok{geom_point}\NormalTok{() }\OperatorTok{+}
\StringTok{  }\KeywordTok{ggtitle}\NormalTok{(}\StringTok{"Condition factor of glass eel vs week of catchment"}\NormalTok{) }
\NormalTok{plot_condfact_pigm}
\end{Highlighting}
\end{Shaded}

\includegraphics{Data_Exploration_Thesis_files/figure-latex/unnamed-chunk-8-3.pdf}

\begin{Shaded}
\begin{Highlighting}[]
\CommentTok{# Nu de conditiefactor duidelijker weergeven, maar nu enkel voor de al verwerkte glasalen}

\NormalTok{nr_locatie_methode <-}\StringTok{ }\KeywordTok{select}\NormalTok{(maaginhoud2, nr, Locatie_A, Locatie_Oever, Methode)}
\NormalTok{nr_pigmentation_condfact <-}\StringTok{ }\KeywordTok{select}\NormalTok{(data_inbo, nr, pigmentation.stage, condition.factor)}

\NormalTok{maaginhoud6 <-}\StringTok{ }\KeywordTok{select}\NormalTok{(maaginhoud, nr, datum, Gewicht.Glasaal, Gewicht.maag.darmstelsel)}
\NormalTok{maaginhoud6 <-}\StringTok{ }\KeywordTok{mutate}\NormalTok{(maaginhoud6, }\DataTypeTok{ratio_visceral_total =}\NormalTok{ Gewicht.maag.darmstelsel }\OperatorTok{/}\StringTok{ }\NormalTok{Gewicht.Glasaal)}
\NormalTok{maaginhoud6_wo_outlier <-}\StringTok{ }\KeywordTok{filter}\NormalTok{(maaginhoud6, ratio_visceral_total }\OperatorTok{<}\StringTok{ }\DecValTok{1}\NormalTok{) }\CommentTok{# reden: zie hieronder}
\NormalTok{maaginhoud6_wo_outlier}\OperatorTok{$}\NormalTok{datum <-}\StringTok{ }\KeywordTok{strptime}\NormalTok{(}\KeywordTok{as.character}\NormalTok{(maaginhoud6_wo_outlier}\OperatorTok{$}\NormalTok{datum), }\StringTok{"%d/%m/%Y"}\NormalTok{)}

\NormalTok{maaginhoud7 <-}\StringTok{ }\KeywordTok{left_join}\NormalTok{(maaginhoud6_wo_outlier, nr_locatie_methode, }\DataTypeTok{by=}\StringTok{'nr'}\NormalTok{)}
\NormalTok{maaginhoud7 <-}\StringTok{ }\KeywordTok{left_join}\NormalTok{(maaginhoud7, nr_pigmentation_condfact, }\DataTypeTok{by=}\StringTok{'nr'}\NormalTok{)}

\NormalTok{plot_condfact_location_}\DecValTok{2}\NormalTok{ <-}\StringTok{ }\KeywordTok{ggplot}\NormalTok{(maaginhoud7, }\KeywordTok{aes}\NormalTok{(}\KeywordTok{week}\NormalTok{(}\KeywordTok{as.Date}\NormalTok{(datum)), condition.factor)) }\OperatorTok{+}
\StringTok{  }\KeywordTok{geom_point}\NormalTok{() }\OperatorTok{+}
\StringTok{  }\KeywordTok{ggtitle}\NormalTok{(}\StringTok{"Condition factor vs week of catchment"}\NormalTok{) }\OperatorTok{+}
\StringTok{  }\KeywordTok{facet_grid}\NormalTok{(.}\OperatorTok{~}\NormalTok{Locatie_A) }\OperatorTok{+}
\StringTok{  }\KeywordTok{xlab}\NormalTok{(}\StringTok{"Week van het jaar 2017"}\NormalTok{) }\OperatorTok{+}
\StringTok{  }\KeywordTok{ylab}\NormalTok{(}\StringTok{"Conditiefactor"}\NormalTok{)}
\NormalTok{plot_condfact_location_}\DecValTok{2}
\end{Highlighting}
\end{Shaded}

\begin{verbatim}
## Warning: Removed 3 rows containing missing values (geom_point).
\end{verbatim}

\includegraphics{Data_Exploration_Thesis_files/figure-latex/unnamed-chunk-8-4.pdf}

\begin{Shaded}
\begin{Highlighting}[]
\NormalTok{plot_condfact_method_}\DecValTok{2}\NormalTok{ <-}\StringTok{ }\KeywordTok{ggplot}\NormalTok{(maaginhoud7, }\KeywordTok{aes}\NormalTok{(}\KeywordTok{week}\NormalTok{(}\KeywordTok{as.Date}\NormalTok{(datum)), condition.factor)) }\OperatorTok{+}
\StringTok{  }\KeywordTok{geom_point}\NormalTok{() }\OperatorTok{+}
\StringTok{  }\KeywordTok{ggtitle}\NormalTok{(}\StringTok{"Condition factor vs week of catchment"}\NormalTok{) }\OperatorTok{+}
\StringTok{  }\KeywordTok{facet_grid}\NormalTok{(.}\OperatorTok{~}\NormalTok{Methode) }\OperatorTok{+}
\StringTok{  }\KeywordTok{xlab}\NormalTok{(}\StringTok{"Week van het jaar 2017"}\NormalTok{) }\OperatorTok{+}
\StringTok{  }\KeywordTok{ylab}\NormalTok{(}\StringTok{"Conditiefactor"}\NormalTok{)}
\NormalTok{plot_condfact_method_}\DecValTok{2}
\end{Highlighting}
\end{Shaded}

\begin{verbatim}
## Warning: Removed 3 rows containing missing values (geom_point).
\end{verbatim}

\includegraphics{Data_Exploration_Thesis_files/figure-latex/unnamed-chunk-8-5.pdf}
\#\#\# Number method

\[ \frac{amount~of~food~~item~i}{total~number~of~items~found~in~gut}~~~~ x ~~ 100  \]
Vertelt dit niet wat hetzelfde als de piecharts? Onderstaande code werkt
ook nog niet helemaal\ldots{}

\begin{Shaded}
\begin{Highlighting}[]
\NormalTok{maaginhoud_index1  <-}\StringTok{ }\NormalTok{maaginhoud2[}\DecValTok{1}\OperatorTok{:}\DecValTok{6}\NormalTok{]}
\NormalTok{maaginhoud_temp <-}\StringTok{ }\NormalTok{maaginhoud2[}\DecValTok{7}\OperatorTok{:}\DecValTok{24}\NormalTok{]}
\ControlFlowTok{for}\NormalTok{ (rw }\ControlFlowTok{in} \DecValTok{1}\OperatorTok{:}\KeywordTok{length}\NormalTok{(maaginhoud_temp)) \{ }
\NormalTok{    tmp <-}\StringTok{ }\NormalTok{maaginhoud_temp[rw,]}\OperatorTok{/}\KeywordTok{sum}\NormalTok{(maaginhoud_temp[rw,]) }
    \ControlFlowTok{if}\NormalTok{ (tmp) \{}
\NormalTok{          maaginhoud_index1 <-}\StringTok{ }\KeywordTok{rbind}\NormalTok{(maaginhoud_index1, tmp); }
\NormalTok{    \} }\ControlFlowTok{else}\NormalTok{ \{}
\NormalTok{      maaginhoud_index1 <-}\StringTok{ }\KeywordTok{rbind}\NormalTok{(maaginhoud_index1, maaginhoud_temp[rw,])}
\NormalTok{    \}}
\NormalTok{\} }
\end{Highlighting}
\end{Shaded}

\subsubsection{Ratio visceral organs/total body
mass}\label{ratio-visceral-organstotal-body-mass}

Second, we calculate the ratio of visceral organs on total body mass.Dit
is een index voor de `fulness' van de glasaal.

\[ \frac{weight~of~stomach-gut~system~of~glass~eel~w}{total~body~mass~of~glass~eel~w}~~~~ x ~~ 100  \]

\begin{Shaded}
\begin{Highlighting}[]
\NormalTok{plot_ratio_method <-}\StringTok{ }\KeywordTok{ggplot}\NormalTok{(maaginhoud6, }\KeywordTok{aes}\NormalTok{(}\KeywordTok{week}\NormalTok{(}\KeywordTok{as.Date}\NormalTok{(datum)), ratio_visceral_total)) }\OperatorTok{+}
\StringTok{  }\KeywordTok{geom_point}\NormalTok{() }\OperatorTok{+}
\StringTok{  }\KeywordTok{ggtitle}\NormalTok{(}\StringTok{"Ratio of visceral organs on total body mass of glass eel vs week of catchment"}\NormalTok{) }\OperatorTok{+}
\StringTok{  }\KeywordTok{ylab}\NormalTok{(}\StringTok{"Ratio viscerale organen / Totale lichaamsgewicht glasaal"}\NormalTok{) }\OperatorTok{+}
\StringTok{  }\KeywordTok{xlab}\NormalTok{(}\StringTok{"Week van 2017"}\NormalTok{)}
\NormalTok{plot_ratio_method  }\CommentTok{# Dikke outlier wegwerken}
\end{Highlighting}
\end{Shaded}

\begin{verbatim}
## Warning: Removed 77 rows containing missing values (geom_point).
\end{verbatim}

\includegraphics{Data_Exploration_Thesis_files/figure-latex/unnamed-chunk-10-1.pdf}

\begin{Shaded}
\begin{Highlighting}[]
\NormalTok{plot_ratio_method <-}\StringTok{ }\KeywordTok{ggplot}\NormalTok{(maaginhoud6_wo_outlier, }\KeywordTok{aes}\NormalTok{(}\KeywordTok{week}\NormalTok{(}\KeywordTok{as.Date}\NormalTok{(datum)), ratio_visceral_total)) }\OperatorTok{+}
\StringTok{  }\KeywordTok{geom_point}\NormalTok{() }\OperatorTok{+}
\StringTok{  }\KeywordTok{ggtitle}\NormalTok{(}\StringTok{"Ratio of visceral organs on total body mass of glass eel vs week of catchment"}\NormalTok{) }\OperatorTok{+}
\StringTok{  }\KeywordTok{ylab}\NormalTok{(}\StringTok{"Ratio viscerale organen / Totale lichaamsgewicht glasaal"}\NormalTok{) }\OperatorTok{+}
\StringTok{  }\KeywordTok{xlab}\NormalTok{(}\StringTok{"Week van 2017"}\NormalTok{)}
\NormalTok{plot_ratio_method }
\end{Highlighting}
\end{Shaded}

\begin{verbatim}
## Warning: Removed 2 rows containing missing values (geom_point).
\end{verbatim}

\includegraphics{Data_Exploration_Thesis_files/figure-latex/unnamed-chunk-10-2.pdf}

According to previous plot, glass eel do tend to be more fed at the end
of spring. This is true for all catchment methods \& locations. But is
this really the case?

\begin{Shaded}
\begin{Highlighting}[]
\NormalTok{plot_ratio_location <-}\StringTok{ }\KeywordTok{ggplot}\NormalTok{(maaginhoud7, }\KeywordTok{aes}\NormalTok{(}\KeywordTok{week}\NormalTok{(}\KeywordTok{as.Date}\NormalTok{(datum)), ratio_visceral_total)) }\OperatorTok{+}
\StringTok{  }\KeywordTok{geom_point}\NormalTok{() }\OperatorTok{+}
\StringTok{  }\KeywordTok{ggtitle}\NormalTok{(}\StringTok{"Ratio of visceral organs on total body mass of glass eel vs week of catchment"}\NormalTok{) }\OperatorTok{+}
\StringTok{  }\KeywordTok{facet_grid}\NormalTok{(.}\OperatorTok{~}\NormalTok{Locatie_A) }\OperatorTok{+}
\StringTok{  }\KeywordTok{ylab}\NormalTok{(}\StringTok{"Ratio viscerale organen / Totale lichaamsgewicht glasaal"}\NormalTok{) }\OperatorTok{+}
\StringTok{  }\KeywordTok{xlab}\NormalTok{(}\StringTok{"Week van 2017"}\NormalTok{)}
\NormalTok{plot_ratio_location}
\end{Highlighting}
\end{Shaded}

\begin{verbatim}
## Warning: Removed 2 rows containing missing values (geom_point).
\end{verbatim}

\includegraphics{Data_Exploration_Thesis_files/figure-latex/unnamed-chunk-11-1.pdf}

\begin{Shaded}
\begin{Highlighting}[]
\NormalTok{plot_ratio_method <-}\StringTok{ }\KeywordTok{ggplot}\NormalTok{(maaginhoud7, }\KeywordTok{aes}\NormalTok{(}\KeywordTok{week}\NormalTok{(}\KeywordTok{as.Date}\NormalTok{(datum)), ratio_visceral_total)) }\OperatorTok{+}
\StringTok{  }\KeywordTok{geom_point}\NormalTok{() }\OperatorTok{+}
\StringTok{  }\KeywordTok{ggtitle}\NormalTok{(}\StringTok{"Ratio of visceral organs on total body mass of glass eel vs week of catchment"}\NormalTok{) }\OperatorTok{+}
\StringTok{  }\KeywordTok{facet_grid}\NormalTok{(.}\OperatorTok{~}\NormalTok{Methode) }\OperatorTok{+}
\StringTok{  }\KeywordTok{ylab}\NormalTok{(}\StringTok{"Ratio viscerale organen / Totale lichaamsgewicht glasaal"}\NormalTok{) }\OperatorTok{+}
\StringTok{  }\KeywordTok{xlab}\NormalTok{(}\StringTok{"Week van 2017"}\NormalTok{)}
\NormalTok{plot_ratio_method}
\end{Highlighting}
\end{Shaded}

\begin{verbatim}
## Warning: Removed 2 rows containing missing values (geom_point).
\end{verbatim}

\includegraphics{Data_Exploration_Thesis_files/figure-latex/unnamed-chunk-11-2.pdf}

\begin{Shaded}
\begin{Highlighting}[]
\NormalTok{plot_ratio_location2 <-}\StringTok{ }\KeywordTok{ggplot}\NormalTok{(maaginhoud7, }\KeywordTok{aes}\NormalTok{(}\KeywordTok{week}\NormalTok{(}\KeywordTok{as.Date}\NormalTok{(datum)), ratio_visceral_total)) }\OperatorTok{+}
\StringTok{  }\KeywordTok{geom_point}\NormalTok{() }\OperatorTok{+}
\StringTok{  }\KeywordTok{ggtitle}\NormalTok{(}\StringTok{"Ratio of visceral organs on total body mass of glass eel vs week of catchment"}\NormalTok{) }\OperatorTok{+}
\StringTok{  }\KeywordTok{facet_grid}\NormalTok{(.}\OperatorTok{~}\NormalTok{Locatie_Oever) }\OperatorTok{+}
\StringTok{  }\KeywordTok{ylab}\NormalTok{(}\StringTok{"Ratio viscerale organen / Totale lichaamsgewicht glasaal"}\NormalTok{) }\OperatorTok{+}
\StringTok{  }\KeywordTok{xlab}\NormalTok{(}\StringTok{"Week van 2017"}\NormalTok{)}
\NormalTok{plot_ratio_location2}
\end{Highlighting}
\end{Shaded}

\begin{verbatim}
## Warning: Removed 2 rows containing missing values (geom_point).
\end{verbatim}

\includegraphics{Data_Exploration_Thesis_files/figure-latex/unnamed-chunk-11-3.pdf}

\begin{Shaded}
\begin{Highlighting}[]
\NormalTok{plot_ratio_pigm_methode <-}\StringTok{ }\KeywordTok{ggplot}\NormalTok{(maaginhoud7, }\KeywordTok{aes}\NormalTok{(pigmentation.stage, ratio_visceral_total, }\DataTypeTok{fill =}\NormalTok{ Methode)) }\OperatorTok{+}
\StringTok{  }\KeywordTok{geom_boxplot}\NormalTok{() }\OperatorTok{+}
\StringTok{  }\KeywordTok{ggtitle}\NormalTok{(}\StringTok{"Ratio of visceral organs on total body mass of glass eel vs pigmentation stage"}\NormalTok{) }\OperatorTok{+}
\StringTok{  }\KeywordTok{ylab}\NormalTok{(}\StringTok{"Ratio viscerale organen / Totale lichaamsgewicht glasaal"}\NormalTok{) }\OperatorTok{+}
\StringTok{  }\KeywordTok{xlab}\NormalTok{(}\StringTok{"Pigmentatiestadium"}\NormalTok{) }\OperatorTok{+}
\StringTok{  }\KeywordTok{scale_fill_manual}\NormalTok{(}\DataTypeTok{values=}\KeywordTok{wes_palette}\NormalTok{(}\DataTypeTok{n=}\DecValTok{3}\NormalTok{, }\DataTypeTok{name=}\StringTok{"Moonrise2"}\NormalTok{))  }
\NormalTok{plot_ratio_pigm_methode}
\end{Highlighting}
\end{Shaded}

\includegraphics{Data_Exploration_Thesis_files/figure-latex/unnamed-chunk-11-4.pdf}

\begin{Shaded}
\begin{Highlighting}[]
\NormalTok{plot_ratio_pigm_locatie <-}\StringTok{ }\KeywordTok{ggplot}\NormalTok{(maaginhoud7, }\KeywordTok{aes}\NormalTok{(pigmentation.stage, ratio_visceral_total, }\DataTypeTok{fill =}\NormalTok{ Locatie_A)) }\OperatorTok{+}
\StringTok{  }\KeywordTok{geom_boxplot}\NormalTok{() }\OperatorTok{+}
\StringTok{  }\KeywordTok{ggtitle}\NormalTok{(}\StringTok{"Ratio of visceral organs on total body mass of glass eel vs pigmentation stage"}\NormalTok{) }\OperatorTok{+}
\StringTok{  }\KeywordTok{scale_fill_manual}\NormalTok{(}\DataTypeTok{values=}\KeywordTok{wes_palette}\NormalTok{(}\DataTypeTok{n=}\DecValTok{4}\NormalTok{, }\DataTypeTok{name=}\StringTok{"Moonrise2"}\NormalTok{)) }\OperatorTok{+}\StringTok{ }
\StringTok{  }\KeywordTok{ylab}\NormalTok{(}\StringTok{"Ratio viscerale organen / Totale lichaamsgewicht glasaal"}\NormalTok{) }\OperatorTok{+}
\StringTok{  }\KeywordTok{xlab}\NormalTok{(}\StringTok{"Pigmentatiestadium"}\NormalTok{)}
\NormalTok{plot_ratio_pigm_locatie}
\end{Highlighting}
\end{Shaded}

\includegraphics{Data_Exploration_Thesis_files/figure-latex/unnamed-chunk-11-5.pdf}

\begin{Shaded}
\begin{Highlighting}[]
\NormalTok{plot_ratio_pigm_methode <-}\StringTok{ }\KeywordTok{ggplot}\NormalTok{(maaginhoud7, }\KeywordTok{aes}\NormalTok{(pigmentation.stage, ratio_visceral_total, }\DataTypeTok{fill =}\NormalTok{ Methode)) }\OperatorTok{+}
\StringTok{  }\KeywordTok{geom_boxplot}\NormalTok{() }\OperatorTok{+}
\StringTok{  }\KeywordTok{scale_fill_manual}\NormalTok{(}\DataTypeTok{values=}\KeywordTok{wes_palette}\NormalTok{(}\DataTypeTok{n=}\DecValTok{3}\NormalTok{, }\DataTypeTok{name=}\StringTok{"Moonrise2"}\NormalTok{)) }\OperatorTok{+}\StringTok{ }
\StringTok{  }\KeywordTok{ggtitle}\NormalTok{(}\StringTok{"Ratio of visceral organs on total body mass of glass eel vs pigmentation stage"}\NormalTok{) }\OperatorTok{+}
\StringTok{  }\KeywordTok{facet_grid}\NormalTok{(.}\OperatorTok{~}\NormalTok{Locatie_A) }\OperatorTok{+}
\StringTok{  }\KeywordTok{ylab}\NormalTok{(}\StringTok{"Ratio viscerale organen / Totale lichaamsgewicht glasaal"}\NormalTok{) }\OperatorTok{+}
\StringTok{  }\KeywordTok{xlab}\NormalTok{(}\StringTok{"Pigmentatiestadium"}\NormalTok{)}
\NormalTok{plot_ratio_pigm_methode}
\end{Highlighting}
\end{Shaded}

\includegraphics{Data_Exploration_Thesis_files/figure-latex/unnamed-chunk-11-6.pdf}

\begin{Shaded}
\begin{Highlighting}[]
\NormalTok{plot_ratio_tijd_methode <-}\StringTok{ }\KeywordTok{ggplot}\NormalTok{(maaginhoud7, }\KeywordTok{aes}\NormalTok{(}\KeywordTok{week}\NormalTok{(datum), ratio_visceral_total, }\DataTypeTok{fill =}\NormalTok{ Methode)) }\OperatorTok{+}
\StringTok{  }\KeywordTok{geom_boxplot}\NormalTok{() }\OperatorTok{+}
\StringTok{  }\KeywordTok{ggtitle}\NormalTok{(}\StringTok{"Ratio of visceral organs on total body mass of glass eel vs time"}\NormalTok{) }\OperatorTok{+}
\StringTok{  }\KeywordTok{scale_fill_manual}\NormalTok{(}\DataTypeTok{values=}\KeywordTok{wes_palette}\NormalTok{(}\DataTypeTok{n=}\DecValTok{4}\NormalTok{, }\DataTypeTok{name=}\StringTok{"Moonrise2"}\NormalTok{)) }\OperatorTok{+}\StringTok{ }
\StringTok{  }\KeywordTok{facet_grid}\NormalTok{(.}\OperatorTok{~}\NormalTok{Locatie_A) }\OperatorTok{+}
\StringTok{  }\KeywordTok{ylab}\NormalTok{(}\StringTok{"Ratio viscerale organen / Totale lichaamsgewicht glasaal"}\NormalTok{) }\OperatorTok{+}
\StringTok{  }\KeywordTok{xlab}\NormalTok{(}\StringTok{"Week van 2017"}\NormalTok{)}
  \CommentTok{#facet_zoom( x = Locatie_A == "Pompgemaal")}
\NormalTok{plot_ratio_tijd_methode}
\end{Highlighting}
\end{Shaded}

\begin{verbatim}
## Warning: Removed 2 rows containing missing values (stat_boxplot).
\end{verbatim}

\includegraphics{Data_Exploration_Thesis_files/figure-latex/unnamed-chunk-11-7.pdf}

And indeed, it is the case across methods and locations.However, data
from the pompgemaal is probably most reliable, since not all
pigmentation stages were caught at other locations, and only substrates
were used to catch glass eels.

\subsection{Diversity Indices}\label{diversity-indices}

As in Charlottes' bachelorproef, we could calculate several diversity
indices. However, these are clasiccally used in ecological community
turnover studies. Are these really applicable?

\begin{Shaded}
\begin{Highlighting}[]
\KeywordTok{library}\NormalTok{(vegan)}
\end{Highlighting}
\end{Shaded}

\begin{verbatim}
## Warning: package 'vegan' was built under R version 3.4.4
\end{verbatim}

\begin{verbatim}
## Loading required package: permute
\end{verbatim}

\begin{verbatim}
## Warning: package 'permute' was built under R version 3.4.4
\end{verbatim}

\begin{verbatim}
## Loading required package: lattice
\end{verbatim}

\begin{verbatim}
## Warning: package 'lattice' was built under R version 3.4.4
\end{verbatim}

\begin{verbatim}
## This is vegan 2.5-4
\end{verbatim}

\begin{Shaded}
\begin{Highlighting}[]
\NormalTok{maaginhoud_index2 <-}\StringTok{ }\KeywordTok{select}\NormalTok{(maaginhoud2, nr, Microplastics.Contaminatie.vezels..katoen..}\OperatorTok{:}\NormalTok{Plantae.sp.)}
\NormalTok{maaginhoud_index2 <-}\StringTok{ }\KeywordTok{remove_empty}\NormalTok{(maaginhoud_index2, }\StringTok{'rows'}\NormalTok{)}


\NormalTok{maaginhoud7 <-}\StringTok{ }\KeywordTok{mutate}\NormalTok{(maaginhoud7, }\DataTypeTok{Shannon =} \KeywordTok{diversity}\NormalTok{(maaginhoud_index2, }\DataTypeTok{index=}\StringTok{"shannon"}\NormalTok{, }\DataTypeTok{MARGIN=}\DecValTok{1}\NormalTok{, }\DataTypeTok{base=}\KeywordTok{exp}\NormalTok{(}\DecValTok{1}\NormalTok{)))}
\NormalTok{maaginhoud7 <-}\StringTok{ }\KeywordTok{mutate}\NormalTok{(maaginhoud7, }\DataTypeTok{Simpson =} \KeywordTok{diversity}\NormalTok{(maaginhoud_index2, }\DataTypeTok{index=}\StringTok{"simpson"}\NormalTok{, }\DataTypeTok{MARGIN=}\DecValTok{1}\NormalTok{, }\DataTypeTok{base=}\KeywordTok{exp}\NormalTok{(}\DecValTok{1}\NormalTok{)))}

\NormalTok{plot_Shannon_tijd_methode <-}\StringTok{ }\KeywordTok{ggplot}\NormalTok{(maaginhoud7, }\KeywordTok{aes}\NormalTok{(}\KeywordTok{week}\NormalTok{(datum), Shannon, }\DataTypeTok{color =}\NormalTok{ Methode)) }\OperatorTok{+}
\StringTok{  }\KeywordTok{geom_point}\NormalTok{() }\OperatorTok{+}
\StringTok{  }\KeywordTok{ggtitle}\NormalTok{(}\StringTok{"Shannon-Index vs time"}\NormalTok{) }\OperatorTok{+}
\StringTok{  }\KeywordTok{scale_fill_manual}\NormalTok{(}\DataTypeTok{values=}\KeywordTok{wes_palette}\NormalTok{(}\DataTypeTok{n=}\DecValTok{4}\NormalTok{, }\DataTypeTok{name=}\StringTok{"Moonrise2"}\NormalTok{)) }\OperatorTok{+}\StringTok{ }
\StringTok{  }\KeywordTok{facet_grid}\NormalTok{(.}\OperatorTok{~}\NormalTok{Locatie_A) }\OperatorTok{+}
\StringTok{  }\KeywordTok{ylab}\NormalTok{(}\StringTok{"Shannon-Index"}\NormalTok{) }\OperatorTok{+}
\StringTok{  }\KeywordTok{xlab}\NormalTok{(}\StringTok{"Week van 2017"}\NormalTok{)}
  \CommentTok{#facet_zoom( x = Locatie_A == "Pompgemaal")}
\NormalTok{plot_Shannon_tijd_methode}
\end{Highlighting}
\end{Shaded}

\begin{verbatim}
## Warning: Removed 2 rows containing missing values (geom_point).
\end{verbatim}

\includegraphics{Data_Exploration_Thesis_files/figure-latex/unnamed-chunk-12-1.pdf}

\begin{Shaded}
\begin{Highlighting}[]
\NormalTok{plot_Shannon_pigm_methode <-}\StringTok{ }\KeywordTok{ggplot}\NormalTok{(maaginhoud7, }\KeywordTok{aes}\NormalTok{(pigmentation.stage, Shannon, }\DataTypeTok{color =}\NormalTok{ Methode)) }\OperatorTok{+}
\StringTok{  }\KeywordTok{geom_point}\NormalTok{() }\OperatorTok{+}
\StringTok{  }\KeywordTok{ggtitle}\NormalTok{(}\StringTok{"Shannon-Index vs pigmentation stage"}\NormalTok{) }\OperatorTok{+}
\StringTok{  }\KeywordTok{scale_fill_manual}\NormalTok{(}\DataTypeTok{values=}\KeywordTok{wes_palette}\NormalTok{(}\DataTypeTok{n=}\DecValTok{4}\NormalTok{, }\DataTypeTok{name=}\StringTok{"Moonrise2"}\NormalTok{)) }\OperatorTok{+}\StringTok{ }
\StringTok{  }\KeywordTok{facet_grid}\NormalTok{(.}\OperatorTok{~}\NormalTok{Locatie_A) }\OperatorTok{+}
\StringTok{  }\KeywordTok{ylab}\NormalTok{(}\StringTok{"Shannon-Index"}\NormalTok{) }\OperatorTok{+}
\StringTok{  }\KeywordTok{xlab}\NormalTok{(}\StringTok{"Pigmentation Stage"}\NormalTok{)}
  \CommentTok{#facet_zoom( x = Locatie_A == "Pompgemaal")}
\NormalTok{plot_Shannon_pigm_methode}
\end{Highlighting}
\end{Shaded}

\includegraphics{Data_Exploration_Thesis_files/figure-latex/unnamed-chunk-12-2.pdf}

\begin{Shaded}
\begin{Highlighting}[]
\NormalTok{plot_Simpson_tijd_methode <-}\StringTok{ }\KeywordTok{ggplot}\NormalTok{(maaginhoud7, }\KeywordTok{aes}\NormalTok{(}\KeywordTok{week}\NormalTok{(datum), Simpson, }\DataTypeTok{color =}\NormalTok{ Methode)) }\OperatorTok{+}
\StringTok{  }\KeywordTok{geom_point}\NormalTok{() }\OperatorTok{+}
\StringTok{  }\KeywordTok{ggtitle}\NormalTok{(}\StringTok{"Simpson-Index vs time"}\NormalTok{) }\OperatorTok{+}
\StringTok{  }\KeywordTok{scale_fill_manual}\NormalTok{(}\DataTypeTok{values=}\KeywordTok{wes_palette}\NormalTok{(}\DataTypeTok{n=}\DecValTok{4}\NormalTok{, }\DataTypeTok{name=}\StringTok{"Moonrise2"}\NormalTok{)) }\OperatorTok{+}\StringTok{ }
\StringTok{  }\KeywordTok{facet_grid}\NormalTok{(.}\OperatorTok{~}\NormalTok{Locatie_A) }\OperatorTok{+}
\StringTok{  }\KeywordTok{ylab}\NormalTok{(}\StringTok{"Simpson-Index"}\NormalTok{) }\OperatorTok{+}
\StringTok{  }\KeywordTok{xlab}\NormalTok{(}\StringTok{"Week van 2017"}\NormalTok{)}
  \CommentTok{#facet_zoom( x = Locatie_A == "Pompgemaal")}
\NormalTok{plot_Simpson_tijd_methode}
\end{Highlighting}
\end{Shaded}

\begin{verbatim}
## Warning: Removed 2 rows containing missing values (geom_point).
\end{verbatim}

\includegraphics{Data_Exploration_Thesis_files/figure-latex/unnamed-chunk-12-3.pdf}

\begin{Shaded}
\begin{Highlighting}[]
\NormalTok{plot_Simpson_pigm_methode <-}\StringTok{ }\KeywordTok{ggplot}\NormalTok{(maaginhoud7, }\KeywordTok{aes}\NormalTok{(pigmentation.stage, Simpson, }\DataTypeTok{color =}\NormalTok{ Methode)) }\OperatorTok{+}
\StringTok{  }\KeywordTok{geom_point}\NormalTok{() }\OperatorTok{+}
\StringTok{  }\KeywordTok{ggtitle}\NormalTok{(}\StringTok{"SImpson-Index vs pigmentation stage"}\NormalTok{) }\OperatorTok{+}
\StringTok{  }\KeywordTok{scale_fill_manual}\NormalTok{(}\DataTypeTok{values=}\KeywordTok{wes_palette}\NormalTok{(}\DataTypeTok{n=}\DecValTok{4}\NormalTok{, }\DataTypeTok{name=}\StringTok{"Moonrise2"}\NormalTok{)) }\OperatorTok{+}\StringTok{ }
\StringTok{  }\KeywordTok{facet_grid}\NormalTok{(.}\OperatorTok{~}\NormalTok{Locatie_A) }\OperatorTok{+}
\StringTok{  }\KeywordTok{ylab}\NormalTok{(}\StringTok{"Simpson-Index"}\NormalTok{) }\OperatorTok{+}
\StringTok{  }\KeywordTok{xlab}\NormalTok{(}\StringTok{"Pigmentation Stage"}\NormalTok{)}
  \CommentTok{#facet_zoom( x = Locatie_A == "Pompgemaal")}
\NormalTok{plot_Simpson_pigm_methode}
\end{Highlighting}
\end{Shaded}

\includegraphics{Data_Exploration_Thesis_files/figure-latex/unnamed-chunk-12-4.pdf}

\url{http://spatialecology.weebly.com/r-code--data/category/plotting}

\begin{Shaded}
\begin{Highlighting}[]
\NormalTok{maaginhoud8 <-}\StringTok{ }\KeywordTok{select}\NormalTok{(maaginhoud2, Microplastics.Contaminatie.vezels..katoen..}\OperatorTok{:}\NormalTok{Plantae.sp.)}

\NormalTok{accurve<-}\KeywordTok{specaccum}\NormalTok{(maaginhoud8, }\DataTypeTok{method=}\StringTok{"random"}\NormalTok{, }\DataTypeTok{permutations=}\DecValTok{100}\NormalTok{)}
\CommentTok{#plot(accurve$sites, accurve$richness, xlab="Number of glass eel analysed", ylab="Species Richness")}
\end{Highlighting}
\end{Shaded}

\section{PCA}\label{pca}

PCA as exploration of the stomach analysis data

\url{https://www.datacamp.com/community/tutorials/pca-analysis-r}

\begin{Shaded}
\begin{Highlighting}[]
\NormalTok{maaginhoud2.pca <-}\StringTok{ }\KeywordTok{prcomp}\NormalTok{(}\KeywordTok{na.omit}\NormalTok{(maaginhoud2[}\DecValTok{7}\OperatorTok{:}\DecValTok{24}\NormalTok{]), }\DataTypeTok{center =} \OtherTok{TRUE}\NormalTok{, }\DataTypeTok{scale. =} \OtherTok{TRUE}\NormalTok{)}
\KeywordTok{str}\NormalTok{(maaginhoud2.pca)}
\KeywordTok{ggbiplot}\NormalTok{(maaginhoud2.pca)}
\end{Highlighting}
\end{Shaded}


\end{document}
